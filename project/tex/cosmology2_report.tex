\documentclass{aa_stuff/aa} 

\usepackage{overhead}
\usepackage{newcommands}

%======================================================
%
%   Introduction stuff
%
%======================================================


\begin{document}
\title{An investigation into the power spectrums of matter and radiation} 

\author{
Johan Mylius Kroken
\inst{1,2}
}
\institute{Institute of Theoretical Astrophysics (ITA), University of Oslo, Norway
\and
Center for Computing in Science Education (CCSE), University of Oslo, Norway}
\titlerunning{Investigating power spectrums}
\authorrunning{Mylius Kroken} 
\date{\today   \quad - \quad GitHub repository link: \url{https://github.com/Johanmkr/AST5220/tree/main/project}}  
\abstract{
    The cosmic microwave background radiation can be seen across the entire sky, and can provide great insight into the nature of several cosmological phenomena. It is therefore of interest to investigate this theoretically, which is the main aim of this article. We consider linear perturbation to the FLRW cosmology in conformal Newtonian gauge in order to create a pipeline for predicting the power spectra of matter and anisotropies to the CMB. This is done by first calculating the background cosmology, and recombination epoch, including neutrinos. We then ignore neutrinos, polarisation and reionisation when evolving the perturbation equations in time, in order to solve for the power spectra which can be compared with observations. Ignoring these effects yield a significant discrepancy in the result for small scale modes, but provide enough accuracy for the large scale modes. As a result, we obtaine a pipeline through which we can obtain power spectra for different cosmologies and compare them to data, although the analysis is only done for the $\Lambda$CDM-cosmology using data from ~\cite{Planck2020}.
}

\maketitle
\bibliographystyle{aa_stuff/aa}

\tableofcontents

%======================================================
%
%   Section input
%
%======================================================



\newpage
\section*{Nomenclature}

\subsection*{Constants of nature}
\begin{itemize}
    \item[$G$] Gravitational constant.\\
            $G=6.6743$\stdf{-11}$\unit{m}^3\unit{kg}^{-1}\unit{s}^{-2}.$
    \item[$k_B$] Boltzmann constant.\\
            $k_B = 1.3806\cross10^{-23}\unit{m}^2\unit{kg}\unit{s}^{-2}\unit{K}^{-1}.$
\end{itemize}

\subsection*{Cosmological parameters}
\begin{itemize}
    \item[$H$] - Hubble parameter.
    \item[$H_0$] - Hubble constant \checkthis{fill in stuff}.
    \item[$a\Hp$]  - Scaled Hubble parameter.
\end{itemize}

\subsection*{Density parameters}
Density parameter $\O_X = \rho_X/\rho_c$ where $\rho_X$ is the density and $\rho_c=8\pi G/3H^2$ the critical density. $X$ can take the following values:
\begin{itemize}
    \item[$b$] - Baryons.
    \item[CDM] - Cold dark matter.
    \item[$\gamma$] - Electromagnetic radiation.
    \item[$\nu$] - Neutrinos.
    \item[$k$] - Spatial curvature.
    \item[$\Lambda$] - Cosmological constant.
\end{itemize}
A $0$ in the subscript indicates the present day value. 
\newpage
\section{Introduction}\label{sec:introduction}

The Cosmic Microwave Background radiation is the leftover radiation from the early universe. It is the most ancient light we can observe, having travelled towards us ever since the Universe became transparent. Therefore, it contains a significant amount of information understanding its nature is of utmost importance for us. The main question we will try to answer is: Why are there fluctuations in the CMB? 

In this project we take a closer look at the CMB power spectrum. This is a plot that show the distribution of temperature fluctuations in the CMB across different angular scaler. This is of great significance to us, since the CMB power spectrum is able to reveal information about the cosmological parameters of our universe, such as the various density components and Hubble constant. It is also able to tell us something about the large scale structures of the Universe, and the overall geometry of space itself. Also, which is perhaps the most interesting, it can yield information about the nature of dark energy. 

The overarching aim is to produce a pipeline that allows us to calculate numerically the CMB power spectrum and matter power spectrum given some cosmological model. The steps will be presented chronologically, and we start by setting up the background cosmology in ~\cref{sec:m1}. Here we solve the evolutionary equations for an isotropic and homogeneous universe using the $\Lambda$CDM-model. One particularly important event in the evolution of the CMB is recombination, ultimately leading to photon decoupling. After this, the photons free stream towards us and is what we today see as the CMB. The entire ~\cref{sec:m2} is devoted to this event, and the time right before and right after it. In ~\cref{sec:m3} we take a step away from the isotropic and homogeneous universe and consider perturbations to the metric in conformal Newtonian gauge. The implications these metric perturbations have on the distribution of matter is discussed, and we end up with a set of coupled differential equations that we can solve numerically. The initial condition of these are found by considering the period of inflation in the very early Universe. We learn to evolve all the perturbation, given some initial condition, until today. ~\cref{sec:m4} concerns itself with finding the initial condition in order to compare the computed values with actual observables. This is done by constructing power spectrums; the angular power spectrum of the CMB and the matter power spectrum of the total density contrast. This comparison allow us to theoretically constrain the model parameter.  

We will only consider neutrinos when we solve the background cosmology and recombination epoch. For the rest of the project, neutrinos are ignored. Polarisarion and the epoch of reionisation are ignored throughout. 
\section{Milestone I - Background Cosmology}\label{sec:m1}

According to the cosmological principle, the universe is homogeneous and isotropic on a large scale. Hence, here are no preferred locations nor directions. Furthermore, we may safely assume that the physical laws that govern our local part of the universe is equally valid elsewhere, at larger distances. 

The aim now is to set up the background cosmology, in which the investigation of all interesting phenomena will take place. Setting up the background cosmology is practically equivalent to solving the \textit{Einstein field equation}:
\begin{equation}\label{eq:introduction:einstein_equation}
    G_\munu = 8\pi GT_\munu,
\end{equation}
where $G_\munu$ is the Einstein tensor describing the geometry of spacetime, $G$ is the gravitational constant and $T_\munu$ is the energy and momentum tensor. This equation thus relates the geometry and shape of spacetime itself, to its energy content (matter included). There are many solutions to ~\cref{eq:introduction:einstein_equation}, but we want the solution to govern a Universe that is spatially isotropic and homogeneous, but may evolve in time. The spacetime metric that satisfies this conditions is the \textit{Friedmann-Lemaître-Robertson-Walker metric} (FLRW)d ~\cite[ch. 8]{carroll_2019}.

We will use this metric in order to describe the background universe, how it may evolve in time, and its history. 
Also write about the following:



\subsection{Theory}\label{sec:m1:theory}

\subsubsection{Fundamentals}\label{sec:m1:theory:fundamentals}

    If we assume the universe to be homogeneous and isotropic, the line elements $\d s$ is given by the FLWR-metric, here in polar coordinates ~\cite[eq. 1.1.11]{weinberg2008cosmology}:
    \begin{equation}\label{eq:m1:theory:fundamentals:FLWR_line_element}
        \d s^2 = -\d t^2 + e^{2x}\left[ \frac{\d r^2}{1-kr^2}+r^2\d\theta^2 + r^2\sin^2\theta\d\phi^2 \right],
    \end{equation}
    where we have introduced $x = \ln(a)$ which will be our primary measure of time. 

    We further model all forms of energy in the universe as perfect fluids, only characterised by their rest frame density $\rho$ and isotropic pressure $p$, and an equation of state relating the two:
    \begin{equation}\label{eq:m1:theory:fundamentals:equation_of_state}
        \omega=\frac{\rho}{p}.
    \end{equation}
    By conservation of energy and momentum we must satisfy $\nabla_\mu T^\munu=0$, which results in the following differential equations for the density of each fluid $\rho_i$, from ~\cite{AST5220LectureNotes}:
    \begin{equation}\label{eq:m1:theory:fundamentals:density_diff_eq}
        \dv{\rho_i}{t} +3H\rho_i(1+\omega_i) = 0,
    \end{equation}
    where we have introduced the Hubble parameter $H \equiv\dot{a}/a=\d x/\d t$. The solution to ~\cref{eq:m1:theory:fundamentals:density_diff_eq} is of the form:
    \begin{equation}\label{eq:m1:theory:fundamentals:solution_to_density_diff_eq}
        \rho_i \propto e^{-3(1+\omega_i)x},
    \end{equation}
    where $\omega_\mathrm{M} = 0$ (matter), $\omega_\mathrm{rad}=1/3$ (radiation), $\omega_\Lambda=-1$ (cosmological constant) and $\omega_k=-1/3$ (curvature). 

    With these assumptions, the solution to the Einstein equations, \cref{eq:introduction:einstein_equation} are the Friedmann equations ~\cite[ch. 8.3]{carroll_2019}, the first of which describes the expansion rate of the universe:
    \begin{equation}
        \label{eq:m1:theory:fundamentals:first_friedmann_equation}
        H^2 = \frac{8\pi G}{3}\sum_i\rho_i - kc^2\expe{-2x} \\
    \end{equation}
    and the second describe how this expansion rate changes over time:
    \begin{equation}
        \label{eq:m1:theory:fundamentals:second_friedmann_equation}
        \dv{H}{t}+H^2 = -\frac{4\pi G}{3}\sum_i\left(\rho+\frac{3p}{c^2}\right).
    \end{equation}
    As of now, we are primarily interested in the first Friedmann equation. By introducing the critical density, $\rho_c\equiv2H^2/(8\pi G)$, we define the density parameters $\O_i=\rho_i/\rho_c$. We further define the density $\rho_k\equiv-3kc^2\expe{-2x}/(8\pi G)$,\footnote{This is the ``density of curvature'', but is in fact not a real density. It is called this because its mathematical behaviour is similar to that of the other (real) densities.} which enables us to write ~\cref{eq:m1:theory:fundamentals:first_friedmann_equation} as simply:
    \begin{equation}
        1=\sum_i\O_i,
    \end{equation}`
    where the density $\O_k$ is included in the sum. From ~\cref{eq:m1:theory:fundamentals:solution_to_density_diff_eq} we know the evolution of the densities in time, and if we assume the density values today, $\O_{i0}$, are known (or are free parameters), then ~\cref{eq:m1:theory:fundamentals:first_friedmann_equation} may also be written as:
    \begin{equation}\label{eq:m1:theory:fundamentals:Hubble_equation}
        H = H_0\sqrt{\sum_i\O_{i0}\expe{-3(1+\omega_i)x}},
    \end{equation}
    which is the Hubble equation we will use further.


\subsubsection{Measure of time and space}\label{sec:m1:measure_time_space}
    There are many ways of measuring time in cosmology, and they are often related to spatial quantities. The most common is perhaps the \textit{scale factor} $a$, which describes how the physical size of the universe changes with time. An increasing scale factor signifies an expanding universe and vice versa. Another, computationally more useful way of describing $a$ is through its logarithm $x=\ln a \iff a=e^x$, which is the convention we will stick to eventually.
    
    Another way of measuring the expansion of the Universe is through the \textit{redshift} $z$, which is defined as the change in wavelength of electromagnetic radiation between emitter and observer. Radiation propagates through the Universe, so any expansion (or contraction) would expand (or contract) the wavelength, and this is encapsulated in the redshift $z=\Delta\lambda/\lambda$. It is connected to the scale factor as $1+z=1/a$.
    
    Another, perhaps more familiar, time measure is the \textit{cosmic time} $t$. This is the time\footnote{In seconds, months, years, or any other preferred temporal unit (like the duration of a footbal match $\pm$ added time).} measured by a stationary observer (relative to the expanding universe). The statement: \textit{The Universe is somewhat 13 billion years old}, is given in cosmic time, i.e. the time we experience on Earth. 

    Lastly, there is the \textit{conformal time} $\eta$, defined as $\d\eta = c\d te^{-x}$.\footnote{The $c$ is sometimes omitted. $\d\eta = \d te^{-x}$ has units of $\unit{s}$, but multiplied with $c$ yields the distance traversed by a light ray in this time; which is the particle horizon.} This is a measure of distance (or rather the time it would take a light ray to traverse said distance) between points in space, where the expansion of space in between the points is taken into account. We use it to define the \textit{particle horizon}, which is the horizon generated by the maximal conformal time elapsed since the Big Bang. This is how ``far away'' from the Big Bang any light ray could have propagated (expansion of the Universe included). This horizon expands with time, as we would expect, and this is what we mean by conformal time from now on; the extent of the particle horizon, beyond which there cannot be any causal connection to the Big Bang. Thus, this is effectively the size of the causally connected universe. 

    Let's express this mathematically, starting with the cosmic time:
    \begin{equation}\label{eq:m1:theory:measures:cosmic_time}
        t = \int_0^a\frac{\d a}{aH} = \int_{-\infty}^{x}\frac{\d x}{H}.
    \end{equation}
    Using the definition of conformal time, we have:
    \begin{equation}\label{eq:m1:theory:measures:conformal_time}
        \eta = \int_0^a\frac{c\d a}{a^2H} = \int_{-\infty}^{x}\frac{c\d x}{e^xH} \equiv \int_{-\infty}^{x}\frac{c\d x}{\Hp},
    \end{equation}
    where $\Hp=e^xH$ is defined as the \textit{conformal Hubble parameter}. We may then define the \textit{comoving distance}, $\chi$, as the distance to a point, where we take the expansion of space into account, such that it becomes constant (given no relative motion). In contrast, the proper distance between two points increase as the universe increase, the comoving distance remain constant. It is related to the conformal time, and given by:
    \begin{equation}
        \label{eq:m1:theory:measures:conformal ditance}
        \chi = \int_1^a\frac{c\d a}{a^2H} = \int_0^x\frac{c\d x}{\Hp} = \eta_0-\eta,
    \end{equation}
    where $\eta_0$ is the conformal time today, and $\eta$ is the conformal time of the time we are measuring distance to.
    The radial coordinate in the FLRW line element, ~\cref{eq:m1:theory:fundamentals:FLWR_line_element}, is given in terms of the comoving distance and the curvature today $\O_{k0}$ as:
    \begin{equation}\label{eq:m1:theory:measures:r_equation_def}
        r = \begin{cases}
            \chi \cdot \frac{\sin\left(\sqrt{\abs{\O_{k0}}}H_0\chi/c\right)}{\sqrt{\abs{\O_{k0}}}H_0\chi/c} \qquad &\O_{k0} < 0 \\
            \chi \qquad &\O_{k0} = 0 \\
            \chi \cdot \frac{\sinh\left(\sqrt{\abs{\O_{k0}}}H_0\chi/c\right)}{\sqrt{\abs{\O_{k0}}}H_0\chi/c} \qquad &\O_{k0} > 0
        \end{cases}
    \end{equation}
    It is then straigtforward to define the angular diameter distance:
    \begin{equation}\label{eq:m1:theory:measures:angular_distance_def}
        d_A = \expe{x}r,
    \end{equation}
    and the luminosity distance:
    \begin{equation}\label{eq:m1:theory:measures:luminosity_distance}
        d_L = \expe{-x}r,
    \end{equation}
    both of which are derived in \cref{app:derivations}. The temporal quantities $\eta$ and $t$ have the following evolutions with $x$:
    \begin{equation}\label{eq:m1:theory:measures:eta_diffeq}
        \dv{\eta}{x} = \frac{c}{\Hp}.
    \end{equation}
    \begin{equation}\label{eq:m1:theory:measures:t_diffeq}
        \dv{t}{x} = \frac{1}{H}.
    \end{equation}
    Both differential equations are easy to solve numerically. Their derivation may also be found in \cref{app:derivations}

\subsubsection{$\Lambda$CDM-model}\label{sec:m1:lambdaCDM}

    In the $\Lambda$CDM model, the universe consists of matter in terms of baryonic matter ($b$) and cold dark matter (CDM), radiation in terms of photons ($\gamma$) and neutrinos ($\nu$) and dark energy in terms of a cosmological constant ($\Lambda$). In addition, we must allow for some curvature ($k$). As a result, the parameters of the model will be the present values of the Hubble rate, $H_0$, the baryon density $\O_{b0}$, the cold dark matter density $\O_{\mathrm{CDM}0}$, photon density $\O_{\gamma0}$, neutrino density $\O_{\nu0}$, dark energy density $\O_{\Lambda0}$, and the curvature density $\O_{k0}$. The present temperature of the cosmic microwave background radiation $T_{\mathrm{CMB}0}$ fixes the radiation density today through:
    \begin{equation}\label{eq:m1:theory:lambdaCDM:radiation_densities}
        \begin{split}
            \O_{\gamma0} &= \frac{16\pi^3G}{90}\cdot\frac{(k_b T_{\mathrm{CMB}0})^4}{\hbar^3c^5H_0^2}, \\
            \O_{\nu0} &= N_\mathrm{eff} \cdot\frac{7}{8} \cdot \left( \frac{4}{3} \right)^{4/3}\cdot \O_{\gamma0}.
        \end{split}
    \end{equation}
    The total radiation density is $\O_\mathrm{rad}=\O_\gamma+\O_\nu$ and the total matter density is $\O_\mathrm{M} = \O_b+\O_\mathrm{CDM}$.

    The Hubble equation from \cref{eq:m1:theory:fundamentals:Hubble_equation} may be redefined in terms of the conformal Hubble parameter $\Hp$ as:
    \begin{equation}\label{eq:m1:lambdaCDM:conformal_Hubble_equation}
        \begin{split}
            \Hp &= H_0\sqrt{U}\\
            U &\equiv \sum_i\O_{i0}\expe{-\alpha_ix}, 
        \end{split}
    \end{equation}
    where we have defined $\alpha_i\equiv(1+3\omega_i)$ and $i\in\{\mathrm{M}, \mathrm{rad}, \Lambda, k\}$. Since we know the values of the various $\omega_i$ it follows that:
    \begin{equation}
        \label{eq:m1:theory:lambdaCDM:alpha_values}
        \begin{split}
            \alpha_\mathrm{M} &= 1\\
            \alpha_\mathrm{rad} &= 2\\
            \alpha_k &= 0\\
            \alpha_\Lambda &= -2
        \end{split}
    \end{equation}

     Given the evolution of the density parameters with time, where the proportionality constant is the present day density, we introduce the \textit{radiation-matter equality}, i.e. the time radiation and matter densities were equal: $\rho_\mathrm{rad}=\rho_\mathrm{M}$. According to \cref{eq:m1:theory:fundamentals:solution_to_density_diff_eq} this can be expressed as:
     \begin{equation}\label{eq:m1:theory:equalities:radiation_matter}
        \begin{split}
            \rho_\mathrm{rad0}e^{-4x} &= \rho_\mathrm{M0}\expe{-3x} \\
            e^x &= \frac{\rho_\mathrm{rad0}}{\rho_\mathrm{M0}} \implies x_\mathrm{rM} = \ln\left(\frac{\O_\mathrm{rad0}}{\O_\mathrm{M0}}\right),
        \end{split}
     \end{equation}
     where $x_\mathrm{rM}$ now denotes the time of radiation-matter equality. 

     Similarly, the \textit{matter-dark energy equality}, where $\rho_\mathrm{M}=\rho_\Lambda$ can be found to be:
     \begin{equation}\label{eq:m1:theory:equalities:matter_dark_energy}
        \begin{split}
            \rho_\Lambda &= \rho_\mathrm{M0}\expe{-3x} \\
            \implies x_{\mathrm{M}\Lambda} &= \frac{1}{3}\ln\left(\frac{\O_\mathrm{M0}}{\O_\Lambda}\right)
        \end{split}
     \end{equation}
     Since $\Hp$ describes the expansion of the Universe, it is fair to say that the acceleration of the Universe is governed by its second derivative, and the acceleration onset may be found from the extremal point in its first derivative. This means that we find the acceleration onset when:
     \begin{equation}\label{eq:m1:theory:accel_onset_condition}
        \dv{\Hp}{x} = 0 \iff \dv{U}{x} = 0.
     \end{equation}
     This follows from ~\cref{eq:m1:lambdaCDM:conformal_Hubble_equation}. For further details on the derivative, see ~\cref{app:sanity}. We assume dark energy is involved in the acceleration of the universe, and thus assume the contribution from radiation is negligible. ~\cref{eq:m1:theory:accel_onset_condition} is thus further reduced to:
     \begin{equation}\label{eq:m1:theory:accel_onset_definiton}
        \begin{split}
            2\O_{\Lambda0}e^{2x} - \O_\mathrm{M0}e^{-x}&=0\\
            \implies x_\mathrm{accel}&=\frac{1}{3}\ln\left(\frac{\O_\mathrm{M0}}{2\O_{\Lambda0}}\right).
        \end{split}
     \end{equation}
    
    The age of the universe today, and the conformal time today can both be found by evaluating the solutions to the differential equations of $t$ and $\eta$ at the present time (where $x=0$). This is done numerically. 

\subsubsection{Analytical solutions and sanity checks}\label{sec:m1:theory:sanity}
    There are several ways we may check that both our workings and numerical implementations are indeed correct. The simplest way is to always ensure that the sum of all density parameters add up to 1, for all times: $\sum_i\O_i=1$. 
    
    If we only consider the most dominant density parameter, that is $\O_i \gg \sum_{j\neq i}\O_j$, we end up with the following analytical expressions for different temporal regimes:
    \begin{equation}
        \label{eq:m1:theory:sanity:first_deriv_sanity}
        \frac{1}{\Hp}\dv{\Hp}{x} \approx - \frac{\alpha_i}{2} = 
        \begin{cases}
            -1 \qquad &\alpha_\mathrm{rad} = 2 \\
            -\frac{1}{2} \qquad &\alpha_\mathrm{M} = 1\\
            1 \qquad &\alpha_\Lambda = -2
        \end{cases}
    \end{equation}
    \begin{equation}
        \label{eq:m1:theory:sanity:second_deriv_sanity}
        \frac{1}{\Hp}\dv[2]{\Hp}{x} \approx \frac{\alpha_i^2}{4} = 
        \begin{cases}
            1 \qquad &\alpha_\mathrm{rad} = 2 \\
            \frac{1}{4} \qquad &\alpha_\mathrm{M} = 1\\
            1 \qquad &\alpha_\Lambda = -2
        \end{cases}
    \end{equation}
    \begin{equation}
        \label{eq:m1:theory:sanity:conformal_hubble}
        \Hp \approx H_0\sqrt{\O_{i0}e^{-\alpha_ix}} = 
        \begin{cases}
            H_0\sqrt{\O_\mathrm{rad0}}e^{-x} \qquad &\alpha_\mathrm{rad} = 2 \\
            H_0\sqrt{\O_\mathrm{M0}}e^{-x/2} \qquad &\alpha_\mathrm{M} = 1\\
            H_0\sqrt{\O_{\Lambda0}}e^x \qquad &\alpha_\Lambda = -2
        \end{cases}
    \end{equation}
    \begin{equation}
        \label{eq:m1:theory:sanity:eta_sanity}
        \frac{\eta\Hp}{c} \approx 
        \begin{cases}
            1 \qquad &\alpha_\mathrm{rad} = 2 \\
            2 \qquad &\alpha_\mathrm{M} = 1\\
            \infty \qquad &\alpha_\Lambda = -2
        \end{cases}
    \end{equation}
    These equations will be useful when making sure that the implementations are correct.\footnote{~\cref{eq:m1:theory:sanity:eta_sanity} is a bit hand-wavy as $\eta$ is really an integral, so assuming a dominant density parameter, means assuming it for the whole existence of the universe, not only the regime we are looking at. We may hence expect these to bee gradually more wrong as the Universe evolves. \TODO{DO I HAVE TIME TO FIX THIS?}} For a thorough derivation, see \cref{app:sanity}.




\subsection{Methods}\label{sec:m1:methods}

\subsubsection{Initial equation}
    We have to consider the time evolution of the density parameters, given some present value, as function of our chosen time parameter, here $x$. The density evolution is implemented as:
    \begin{equation}\label{eq:m1:methods:initial:density_evolution}
        \O_n = \expe{-\alpha_nx}\O_{n0}\Hp_\mathrm{rat}^2
    \end{equation}
    where we have defined the ratio $\Hp_\mathrm{rat} \equiv H_0/\Hp$, and the new index $n$ are all the densitis: $n\in\{b, \mathrm{CDM}, \gamma, \nu, \Lambda, k\}$.

    We also implement functions to solve for the luminosity distance (\cref{eq:m1:theory:measures:luminosity_distance}), angular distance (\cref{eq:m1:theory:measures:angular_distance_def}), and the conformal distance (\cref{eq:m1:theory:measures:conformal ditance}).


\subsubsection{ODEs and Splines}
    The differential equations for $\eta$ (\cref{eq:m1:theory:measures:eta_diffeq}) and $t$ (\cref{eq:m1:theory:measures:t_diffeq}) are solved numerically as ordinary differential equations with the Runge-Kutta 4 as advancement method. The equations are solved for $x\in(-20,5)$. As initial condition we would like $\eta(-\infty)$ which is obviously not possible to calculate, so we pick some very early time and use the analytical approximation in the radiation dominated era (\cref{eq:m1:theory:sanity:eta_sanity}), which yield:
    \begin{equation}\label{eq:m1:methods:odes:eta_initial}
        \eta(x_0) = \frac{c}{\Hp(x_0)}.
    \end{equation}
    Likewise for $t$, the initial condition is:
    \begin{equation}\label{eq:m1:methods:odes:t_initial}
        t(x_0) = \frac{1}{2H(x_0)}.
    \end{equation}
    
    We then proceed by making splines of both $\eta$ and $t$ in order to evaluate accurately for any $x\in(-20,5)$. 


\subsubsection{Model evaluation}
    We evaluate the model by computing the quantities presented in \cref{sec:m1:theory:sanity} and compare with the analytical solutions in different regimes. This will ensure that the model behave as expected.

    Furthermore, we want the model to somewhat resemble reality, we thus use measures of the luminosity distance of supernovas at different redshifts $z$, acquired by \cite{Betoule_2014}. This data is compared to the prediction made by our model. 

    
    In order to constrain the possible values $\O_\mathrm{M}$ and $\O_\Lambda$ we find the $\chi^2$-error between the luminosity distance of the supernovas and the predictions made by our model. The $\O$-s are sampled with Markov-Chain Monte Carlo sampling using the Metropolis-Hastings algorithm. The $\chi^2$-erros is given by:
    \begin{equation}\label{eq:m1:chi2_test_def}
        \chi^2(h, \O_{m0}, \O_{k0}) = \sum_{i=1}^N \frac{(d_L(z, \O_{m0}, \O_{k0}) - d_L^\mathrm{obs}(z_i))^2}{\sigma_i^2}.
    \end{equation}


    
\subsection{Results}\label{sec:m1:results}

\subsubsection{Tests}\label{sec:m1:results:tests}

\begin{figure}
    \includegraphics[width=\linewidth]{testing_omegas.pdf}
    \caption{Omega tests}
    \label{fig:m1:omega_tests}
\end{figure}

\begin{figure}
    \includegraphics[width=\linewidth]{eta_test.pdf}
    \caption{Eta tests}
    \label{fig:m1:eta_tests}
\end{figure}

\begin{figure}
    \includegraphics[width=\linewidth]{Hp_test.pdf}
    \caption{HP tests}
    \label{fig:m1:Hp_tests}
\end{figure}

\subsubsection{Analysis}

\begin{figure}
    \includegraphics[width=\linewidth]{conformal_hubble_factor.pdf}
    \caption{Conformal Hubble factor.}
    \label{fig:m1:conformal_hubble_factor_Hp}
\end{figure}

\begin{figure}
    \includegraphics[width=\linewidth]{cosmic_conformal_time.pdf}
    \caption{cosmic time.}
    \label{fig:m1:cosmic_conformal_time}
\end{figure}

\begin{figure}
    \includegraphics[width=\linewidth]{supernova_data.pdf}
    \caption{Supernova data fitted}
    \label{fig:m1:supernova_data}
\end{figure}

\begin{figure}
    \includegraphics[width=\linewidth]{omega_plane.pdf}
    \caption{one sigma confidence plot}
    \label{fig:m1:omega_planes}
\end{figure}

\begin{figure}
    \includegraphics[width=\linewidth]{posterior_pdf.pdf}
    \caption{posterior pdf.}
    \label{fig:m1:posterior_pdf}
\end{figure}


\section{Recombination History}\label{sec:m2}

The main goal of this section is to investigate the recombination history of the universe. This can be explained as the point in time when photons decouple from the equilibrium of the opaque early universe.  This is known as the \textit{time of last scattering},\footnote{Which is exactly what the name suggests.} and these photons are what we today observe as the CMB. This period of the history of the universe is thus crucial for understanding the CMB. 

We will start by calculating the free \textit{electron fraction} $X_e$, from which we may find the \textit{optical depth} $\tau$. This again enables us to compute the \textit{visibility function}, $g$, and the \textit{sound horizon}, $s$. The latter will be of great importance later. 

Recombination happens because the expansion of the Universe cools it down, making the photons less energetic, which in turn make each interaction in the primordial plasma less energetic. At some point, hydrogen atoms are able to form, reducing the number of free electron, hence reducing photon interactions, until they scatter for the last time. We will determine the time of recombination from the free electron fraction, which indirectly tell us how large portion of the free electrons have (re)-combined.\footnote{As with any (hopefully good) article on the subject, we ought to say that recombination is a funny wording, as this is the first time in the history of the Universe that protons and electrons combine to form hydrogen.} Due to the decrease of free electrons, photons interact less with them. At some point, photons scatter for the last time, and this information is encapsulated in the visibility function. 


\subsection{Theory}\label{sec:m2:theory}
    The optical depth as a function of conformal time is defines as \cite{AST5220LectureNotes}:
    \begin{equation}\label{eq:m2:theory:optical_depth}
        \tau = \int_\eta^{\eta_0} n_e\sigma_\mathrm{T}\expe{-x}\d\eta',
    \end{equation}
    where $n_e$ is the electron density and $\sigma_\mathrm{T}$ is the Thompson cross-section. From this we define the visibility function, $g$:
    \begin{equation}
        \begin{split}
            g &= -\dv{\tau}{\eta}\expe{-\tau} = -\Hp\dv{\tau}{x}\expe{-\tau}\\
            \tilde{g} &\equiv -\dv{\tau}{x}\expe{-\tau} = \frac{g}{\Hp},
        \end{split}
    \end{equation}
    where $\tilde{g}$ is in terms of the preferred time variable, $x$. So far, so good, but in order to find $\tilde{g}$ we need $\tau$, which again require $n_e$, which is not trivial to find, since the electron density changes throughout the evolution of the universe.
    
    \subsubsection{Finding the free electron fraction $X_e$}
    We express the electron density through the free electron fraction $X_e \equiv n_e/n_\mathrm{H} = n_e/n_b$ where we have assumed that the hydrogen make up all the baryons ($n_b=n_\mathrm{H}$). We also ignore the difference between free protons and neutral hydrogen. From \cite{https://doi.org/10.48550/arxiv.astro-ph/0606683} we obtain:
    \begin{equation}
        n_b = \frac{\rho_b}{m_\mathrm{H}} = \frac{\O_b\rho_c}{m_\mathrm{H}}\expe{-3x},
    \end{equation}
    where $m_\mathrm{H}$ is the mass of the hydrogen atom, and $\rho_c$ the critical density today as defined earlier (\FIXME{cite this?}). At early times, before recombination, $X_e \simeq 1$ (\FIXME{why?}), and is in this regime described by the \textit{Saha equation}, from \cite{dodelson2020modern}:
    \begin{equation}\label{eq:m2:theory:Saha_equation}
        \frac{X_e^2}{1-X_e} = \frac{1}{n_b}\left(\frac{m_eT_b}{2\pi}\right)^{3/2}\expe{-\epsilon_0/T_b},
    \end{equation}
    where $\epsilon_0 = 13.6\text{ eV}$ is the ionisation energy of hydrogen. The Saha equation is only a good approximation when $X_e \simeq 1$, thus for $X_e < (1-\xi)$, (where we have to define $\xi$) which corresponds to the period during and after recombination, we have to make use of the more accurate \textit{Peebles equation}. From \cite{https://doi.org/10.48550/arxiv.astro-ph/0606683}:
    \begin{equation}\label{eq:m2:theory:peebles_equation}
        \dv{X_e}{x} = \frac{C_r(T_b)}{H}\left[\beta(T_b)(1-X_e)-n_\mathrm{H}\alpha^{(2)}(T_b)X_e^2\right],
    \end{equation}
    where
    \begin{equation}
        \begin{split}
            C_r(T_b) &= \frac{\Lambda_{2s-1s}+\Lambda_\alpha}{\Lambda_{2s-1s} + \Lambda_\alpha+\beta^{(2)}(T_b)}, \\
            \Lambda_{2s-1s} &= 8.227 \unit{s}^{-1}, \\
            \Lambda_\alpha &= H\frac{(3\epsilon_0)^3}{(8\pi)^2n_{1s}}, \\
            n_{1s} &=(1-X_e)n_\mathrm{H}, \\
            n_\mathrm{H} &= (1-Y_p)\frac{3H_0^2\O_{b0}}{8\pi Gm_\mathrm{H}}\expe{-3x},\\
            \beta^{(2)}(T_b) &= \beta(T_b)\expe{3\epsilon_0/4T_b}, \\
            \beta(T_b) &= \alpha^{(2)}(T_b)\left(\frac{m_eT_b}{2\pi}\right)^{3/2}\expe{-\epsilon_0/T_b}, \\
            \alpha^{(2)}(T_b) &=\frac{64\pi}{\sqrt{27\pi}}\frac{\alpha^2}{m_e^2}\sqrt{\frac{\epsilon_0}{T_b}}\phi_2(T_b), \\
            \phi_2(T_b) &= 0.448\ln\left(\frac{\epsilon_0}{T_b}\right).
        \end{split}
    \end{equation}
    \TODO{Add $\sigma_T$ and $\alpha$ to nomenclature}.

    \TODO{Describe the above equations slightly}

    We find by $X_e$ by solving \cref{eq:m2:theory:Saha_equation} for $X_e > (1-\xi)$ and \cref{eq:m2:theory:peebles_equation} for $X_e < (1-\xi)$. 
    \TODO{Explain why, difficult to integrate Peebles at early times etc. }
\subsection{Methods}\label{sec:m2:methods}

\subsubsection{Computing $X_e$}\label{sec:m2:methods:electron_fraction}
    First things first, we need to compute the free electron fraction $X_e$. We are for the most part not interested in things happening in the future here, so the temporal range of choice will be $x\in[-20,0)$ where $x=0$ is today, and $x=-20$ is sufficiently long ago, so that the range encapsulated effect studied here. In the early Universe, the energies are so high that all baryonic matter is in the form of free electron, $X_e\simeq1$, so we will start by solving the Saha equation, ~\cref{eq:m2:theory:Saha_equation}. We continue to solve equation ~\cref{eq:m2:theory:Saha_equation} as long as $X_e>1-\xi$ where we use $\xi=0.01$.

    If we define:
    \begin{equation}
        K \equiv \frac{1}{n_b}\left(\frac{k_Bm_eT_b}{2\pi\hbar^2}\right)^{3/2}\expe{-\epsilon_0/k_BT_b},
    \end{equation}
    then equaton ~\cref{eq:m2:theory:Saha_equation} takes the form $X_e^2 + KX_e - K = 0$, which is solved as a normal quadratic equation\footnote{$ay^2+by+c=0$ has solutions $$y=\frac{-b\pm\sqrt{b^2-4ac}}{2}.$$}, where $a=1$, $b=K$ and $c=-K$. Since $0\leq X_e\leq1$ we choose the positive solution, given by:
    \begin{equation}\label{eq:m2:methods:sqrt_approx}
        X_e = \frac{-K+\sqrt{K^2+4K}}{2} = \frac{K}{2}\left(-1+\sqrt{1+4K^{-1}}\right)
    \end{equation}
    This solution has the potential to become numerically unstable if the parenthesis is close to zero, i.e. for $K\gg1$. We then make use of the approximation $\sqrt{1+4K^{-1}} \approx 1+(2K^{-1})$ for $\abs{4K^{-1}}\ll1$, which ensures $X_e\simeq1$ for very high temperatures (large $K$).


    We continue to solve the Peebles equation as stated in ~\cref{eq:m2:theory:peebles_equation}, where the r.h.s. is implemented sequentially as ~\cref{eq:m2:theory:peebles_CR}-~\cref{eq:m2:theory:peebles_phi2} in reverse order. The initial condition is the last computed electron fraction above the cut-off: $X_{e0}=\min(X_e>1-\xi)$ as found from the Saha equation. It is solved for the x-range not solved by the Saha equation. 

    Having found $X_e$ for the entire x-range, we compute $n_e$ and spline both results.

\subsubsection{Computing $\tau$ and $\tilde{g}$}\label{sec:m2:methods:tau_and_g}
    With $n_e$ we are able to solve the optical depth as defined in ~\cref{eq:m2:theory:optical_depth_differential}. The inital condition for this equation is that the optical depth today is zero: $\tau(x=0)=0$, meaning we have to solve this backwards in time. This is done by using the negative differential:
    \begin{equation}
        \dv{\tau_\mathrm{rev}}{x_\mathrm{rev}} = -\dv{\tau}{x} = \frac{cn_e\sigma_Te^x}{\Hp},
    \end{equation}
    and solving for positive $x_\mathrm{rev}$: $x_\mathrm{rev}\in[0,20]$. In order to undo this reversal, we map $\tau=-\tau_\mathrm{rev}$ to its corresponding $x=-x_\mathrm{rev}$. Having found $\tau$, we find its derivative by solving equation ~\cref{eq:m2:theory:optical_depth_differential}, and further the find the visibility function from ~\cref{eq:m2:theory:visibility_function} and its derivative from ~\cref{eq:m2:theory:visibility_function_deriv}. All of these four quantities are splines, and their derivatives are obtained numerically.

    In order to solve equation ~\cref{eq:m2:theory:sound_horizon_def} for the sound horizon, we choose initial conditions $s_i = c_{s,i}/{\Hp_i}$ where the subscript $i$ denote a very early time (in our case when $x=-20$). We are then able to solve the differential equation for the sound horizon, ~\cref{eq:m2:theory:sound_horizon_differential}, numerically and then spline the result. 


\subsubsection{Analysis}\label{sec:m2:methods:analysis}
    Having splines for the relevant quantities enables us to compute some important times in the early universe. Firstly, the \textit{last scattering surface}, is the time when most photons scattered for the last time, and decoupled from the plasma. This is not expected to have happened instantly, but recalling that the visibility function $\tilde{g}$ is a probability distribution function for when photons last scattered, we simply use the peak of this function as the definition of the last scattering surface. 

    Further, we want to find a time for when recombination happened, i.e. when free electron was captured by protons to form hydrogen atoms. Thus, this coincides with the reduction of the free electron fraction, and we will use $X_e=0.1$ as the definition for when recombination happened. These numbers can also be computed using only the Saha approximation, for comparison. We also compute the sound horizon at these decouplings: $r_s = s(x_\mathrm{dec})$.

    The last thing we want to compute is the freeze out abundance of free electrons, i.e. the free electron abundance today, which is found by evaluating the spline for $X_e$ at $x=0$.




\subsection{Results}\label{sec:m2:results}


\section{Milestone III}\label{sec:m3}

Some introduction to milestone 3

\subsection{Theory}\label{sec:m3:theory}

\subsubsection{Metric perturbations}
    The perturbed metric in the conformal-Newtonian gauge is given in ~\cite{https://doi.org/10.48550/arxiv.astro-ph/0606683} as:
    \begin{equation}\label{eq:m3:theory:perturbed_metric}
        g_\munu = \begin{pmatrix}
            -(1+2\Psi) & 0 \\
            0 & e^{2x}\delta_{ij}(1+2\Phi)
        \end{pmatrix}
    \end{equation}
    This means that we perturb the FLRW-metric with $\Psi\ll1$ corresponding to the Newtonian potential governing the motion of non-relativistic particles and $\Phi\ll1$ governing the perturbation of the spatial curvature. \footnote{$\Phi$ may also be interpreted as a \textit{local perturbation to the scale factor}, ~\cite{dodelson2020modern}.} The comoving momentum in this spacetime is:
    \begin{equation}
        P^\mu = \left[E(1-\Psi), p^i\frac{1-\Phi}{a}\right].
    \end{equation}
    By considering this momentum, and the geodesic equation in this perturbed spacetime we obtain the following ~\cite[Eqs. 3.62, 3.69, 3.71]{dodelson2020modern}:
    \begin{subequations}\label{eq:m3:theory:geodesic_results}
        \begin{align}
            \dv{x^i}{t} &= \frac{\hat{p}^i}{a}\frac{p}{E}(1-\Phi+\Psi) \label{eq:m3:theory:metric_perturb_dxidt}\\
            \dv{p^i}{t} &= -\left(H+\dv{\Phi}{t}\right)p^i-\frac{E}{a}\pdv{\Phi}{x^i}-\frac{1}{a}\frac{p^i}{E}p^k\pdv{\Phi}{x^k} + \frac{p^2}{aE}\pdv{\Phi}{x^i} \label{eq:m3:theory:metric_perturb_dpidt}\\
            \dv{p}{t} &= -\left(H+\dv{\Phi}{t}\right)p-\frac{E}{a}\hat{p}^i\pdv{\Psi}{x^i} \label{eq:m3:theory:metric_perturb_dpdt}
        \end{align}
    \end{subequations}
    Inserting ~\cref{eq:m3:theory:geodesic_results} into ~\cref{eq:m2:theory:bolzmann_expanded}, and for now assuming $C[f]=0$ yield the \textit{collisionless Bolzmann equations}. Keeping terms to first order only,\footnote{This is justified by the ansatz that deviations away from the equilibrium distribution of radiation in the inhomogeneous universe are of same order as the spacetime perturbations $\Phi$ and $\Psi$, ~\cite{dodelson2020modern}.} yield the collisionless Bolzmann equation: ~\cite[Eq. 3.83]{dodelson2020modern}:
    \begin{equation}\label{eq:m3:theory:collisionless_bolzmann_general}
        \dv{f}{t} = \pdv{f}{t}+\frac{p}{E}\frac{\hat{p}^i}{a}\pdv{f}{x^i}-\left[H+\dv{\Phi}{t} + \frac{E}{ap}\hat{p}^i\pdv{\Psi}{x^i}\right]p\pdv{f}{p}.
    \end{equation}
    Future work consists mainly of evaluating the collision terms for each species and equate it to ~\cref{eq:m3:theory:collisionless_bolzmann_general}

\subsubsection{Fourier space and multipole expansion}
    Consider a function $f(\vec{x},t)$. Its Fourier transform $\FF$ and inverse $\FF^{-1}$ are defined as:
    \begin{equation}\label{eq:m3:theory:fourier_def}
            \FF[f(\vec{x},t)]  \equiv \frac{1}{(2\pi)^{3/2}}\int e^{-i\vec{k}\cdot\vec{x}}f(\vec{x},t)\d^3 x = \tilde{f}(\vec{k},t),
    \end{equation}
    \begin{equation}\label{eq:m3:theory:fourier_inv_def}
        \FF^{-1}[\tilde{f}(\vec{k},t)] \equiv \frac{1}{(2\pi)^{3/2}}\int e^{i\vec{k}\cdot\vec{x}}\tilde{f}(\vec{k},t)\d^3 k = f(\vec{x},t).
    \end{equation}
    It becomes apparent from these definitions that taking the spatial derivative with respect to $\vec{x}$ in real space, is the same as multiplying the function with $i\vec{k}$ in Fourier space. This leads to the following property: $\FF[\nabla f(\vec{x},t)] = i\vec{k}\FF[f(\vec{x},t)]$. This is of major significance when working with partial differential equations (PDEs), where:
    \begin{equation}\label{eq:m3:theory:fourier_pde_tricks}
        \begin{split}
            \FF[\nabla^2f(\vec{x},t)] &= i^2\vec{k}\cdot\vec{k}\FF[f(\vec{x},t)] = -k^2\FF[f(\vec{x},t)]\\
            \FF\left[\dv[n]{f(\vec{x},t)}{t}\right] &= \dv[n]{t}\FF[f(\vec{x},t)].
        \end{split}
    \end{equation} 
    The two equations in ~\cref{eq:m3:theory:fourier_pde_tricks} have the ability of reducing PDEs down to a set of decoupled ODEs. This means that we are able to solve for each mode $k=\abs{\vec{k}}$ independently, which will be of great impact for the equations to come. 

    We will also work with multipole expansions, which are series written as sums of \textit{Legendre polynomials} expanded in $\mu=\cos{\theta}\in[-1,1]$ as:
    \begin{equation}\label{eq:m3:theory:legendre_expansion}
        f(\mu) = \sum_{l=0}^{\infty}f_l\mathcal{P}_l(\mu),
    \end{equation}
    where $\mathcal{P}_l$ is the $l$-th Legendre polynomial. These are orthogonal in such a way that they form a complete basis, enabling us to express any $f(\mu)$ as in ~\cref{eq:m3:theory:legendre_expansion}. The coefficients $f_l$ are the \textit{Legendre multipoles}:
    \begin{equation}\label{eq:m3:theory:legendre_coeff}
        f_l = \frac{2l+1}{2}\int_{-1}^1f(\mu)\mathcal{P}_l(\mu)\d\mu.
    \end{equation}




\subsubsection{Einstein-Boltzmann equations}
    We have two perturbations to the metric, $\Phi(\vec{x}, t)$ to the spatial curvature, and $\Psi(\vec{x},t)$ to the Newtonian potential. We seek to find the effect of these perturbations on baryonic matter, dark energy and radiation, as they ``live'' in a now perturbed spacetime. Let's start by defining the perturbation to the photons, $\T(\vec{x}, \hat{\vec{p}}, t)$, to be the variation of photon temperature around an equilibrium temperature $T^{(0)}$:
    \begin{equation}\label{eq:m3:theory:temperature_perturbation}
        T(\vec{x}, \hat{\vec{p}}, t) = T^{(0)}\left[1+\T(\vec{x}, \hat{\vec{p}}, t)\right].
    \end{equation}
    This is dependent on the location $\vec{x}$ and the direction of propagation $\hat{\vec{p}}$, thus capturing both inhomogeneities and anisotropies. We assume $\T$ to be independent of the momentum magnitude.\footnote{This follows from the fact that the magnitude of the photon momentum is virtually unchanged by the dominant form of interaction, Compton scattering ~\cite{dodelson2020modern}.}
    The collision terms for the photons are governed by Compton scattering. We use the form found in ~\cite[Eq. 5.22]{dodelson2020modern}\TODO{assumptions: ignore polarisation, and angular dep. of thomson cross sec}:
    \begin{equation}\label{eq:m3:theory:photon_collision_term}
        C[f(\vec{p})] = -p^2\pdv{f^{(0)}}{p}n_e\sigma_T[\T_0 - \T(\hat{\vec{p}}) + \hat{\vec{p}}\cdot\vec{v}_b]
    \end{equation}
    where $\T_0$ is the monopole term.\footnote{This is the integral over the photon perturbation at any given point, over all photon directions. It is given by $$\T_0(\vec{x}, t) \equiv \frac{1}{4\pi}\int \d\Omega'\T(\vec{x}, \hat{\vec{p}}', t)$$ where $\O'$ is the solid angle spanned by $\hat{\vec{p}}'$ ~\cite{dodelson2020modern}.} The distribution function for radiation follows the Bose-Einstein distribution function, so we expand $f$ around its zeroth order Bose-Einstein form, ~\cite[Eq. 5.2-5.9]{dodelson2020modern}, using the temperature perturbation in ~\cref{eq:m3:theory:temperature_perturbation}\TODO{Include equation 5.9 in Dodelson?}. This is then inserted into ~\cref{eq:m3:theory:collisionless_bolzmann_general}, which we equate to the collision term in ~\cref{eq:m3:theory:photon_collision_term} in order to obtain the following full Boltzmann equation for radiation:
    \begin{equation}\label{eq:m3:theory:boltzmann_equation_radiation_full}
        \dv{\T}{t} + \frac{\hat{p}^i}{a}\pdv{\T}{x^i} + \dv{\Phi}{t} + \frac{\hat{p}^i}{a}\pdv{\Psi}{x^i} = n_e\sigma_T\left[\T_0-\T+\vec{\hat{p}}\cdot\vec{v}_b\right]
    \end{equation} 


    from ~\cite[Eq. 3.76]{dodelson2020modern}:
    \begin{equation}\label{eq:m3:theory:boltzmann_for_massive_particles}
        \dv{f}{t} = \pdv{f}{t}+\frac{p}{E}\frac{\hat{p}^i}{a}\pdv{f}{x^i} - \left[H+\dot{\Psi}+\frac{E}{ap}\hat{p}^i\Psi_i\right]p\pdv{f}{p}
    \end{equation}


     

    \begin{subequations}
        \begin{align}
            \dot{\T} &= -ik\mu(\T+\Psi)-\dot{\Phi}-\dot{\tau}\left[\T_0-\T+i\mu v_b-\frac{\mathcal{P}_2\T_2}{2}\right],\\
            \dot{\delta}_\cdm &= -3\dot{\Phi}+kv_\cdm \\
            \dot{v}_\cdm &= -k\Psi-\Hp v_\cdm \\
        \end{align}
    \end{subequations}

    
    Photon temperature multipoles
    \begin{subequations}\label{eq:m3:theory:photon_temp_multipoles}
        \begin{align}
            \T_0' &= -\frac{ck}{\Hp}\T_1-\Phi',\label{eq:m3:theory:photon_monopole}\\
            \T_1' &= \frac{ck}{3\Hp}\T_0 - \frac{2ck}{3\Hp}\T_2 +\frac{ck}{3\Hp}\Psi+\tau'\left[\T_1+\frac{1}{3}v_b\right],\label{eq:m3:theory:photon_dipole}\\
            \T_l &= \begin{cases}
                \frac{lck\T_{l-1}}{(2l+1)\Hp} -\frac{(l+1)ck\T_{l+1}}{(2l+1)\Hp} +\tau'\left[\T_l-\frac{\T_2}{10}\delta_{l,2}\right] , \quad &l\geq2\\
                \\
                \frac{ck\T_{l-1}}{\Hp} - c\frac{(l+1)\T_l}{\Hp\eta}+\tau'\T_l, &l=l_\mathrm{max}
            \end{cases}\label{eq:m3:theory:photon_multipole}
        \end{align}
    \end{subequations}


    
    Cold dark matter and baryons
    \begin{subequations}\label{eq:m3:theory:cdm_baryon_diffeqs}
        \begin{align}
            \delta'_\cdm &= \frac{ck}{\Hp}v_\cdm-3\Phi', \label{eq:m3:theory:delta_cdm}\\
            v'_\cdm &= -v_\cdm-\frac{ck}{\Hp}\Psi, \label{eq:m3:theory:v_cdm}\\
            \delta'_b &= \frac{ck}{\Hp}v_b - 3\Phi', \label{eq:m3:theory:delta_b}\\
            v'_b &= -v_b - \frac{ck}{\Hp} + \tau'R^{-1}(3\T_1+v_b) \label{eq:m3:theory:v_b}
        \end{align}
    \end{subequations}
    where $R$ is defined in ~\cref{eq:m2:theory:sound_speed}


    Metric perturbations

    \begin{subequations}\label{eq:m3:theory:metric_perturbations_final}
        \begin{align}
            \Phi' &= \Psi - \frac{c^2k^2}{3\Hp^2}\Phi + \frac{H_0^2}{2\Hp^2}\mathcal{Y}, \label{eq:m3:theory:phi_perturbation}\\
            \Psi &= -\Phi - \frac{12H_0^2}{c^2k^2}\O_\gamma\T_2. \label{eq:m3:theory:psi_perturbation}
        \end{align}
    \end{subequations}
    where $\mathcal{Y} = \O_\cdm\delta_\cdm+\O_b\delta_b+4\O_\gamma\T_0$

\subsubsection{Tight coupling regime}

\subsubsection{Inflation}
\subsubsection{Initial conditions}

\subsection{Methods}\label{sec:m3:methods}

some methods
\subsection{Results and discussion}\label{sec:m3:results}
\section{Milestone IV}\label{sec:m4}

Some introduction to milestone 4

\subsection{Theory}\label{sec:m4:theory}

Some theory
\subsection{Methods}\label{sec:m4:methods}
    First things first, we are going to calculate a lot of spherical Bessel functions, since we need to integrate these across $x$ in finding the transfers function (for every $k$). Thus, it is computationally wise to spline them in advance. Further, we need to calculate the line of sight integral across all the $x$-values in order to obtain the transfer function. We may choose to compute the transfer function direction when integrating the power spectrum, or we may precompute it and then spline the result. We will choose the latter. Lastly, we integrate and compute $C_l$ as functions of $l$. We expect this to be relatively smooth function, so we choose a small finite number of $l$-s for which we perform the integral. 
    The $l$-s we are going to use are the following:

    $l\in$ [2,    3,    4,    5,    6,    7,    8,    10,   12,   15,   
    20,   25,   30,   40,   50,   60,   70,   80,   90,   100,  
    120,  140,  160,  180,  200,  225,  250,  275,  300 \dots 2000],
    where the dots represent increments of 50. 
    \subsubsection{Making Bessel splines}
        We may precompute and spline the spherical Bessel function since we know the minimal and maximum value of its argument. This is, we need to compute for values of $z\equiv k(\eta_0-\eta(x))$ where $\eta_0$ is the current value. From ~\cref{fig:m1:cosmic_conformal_time} we have that $\eta(x) \leq \eta_0$ for $x\leq0$,\footnote{The line of sight integration is from early times until today, so we do not need to think about future value, hence $x\leq0$.} so it becomes apparent that we need to precompute values for $z\in[0, \eta_0k_\mathrm{max}]$ for every $l$. Since the spherical Bessel functions oscillate with a period of approximately $2\pi$, we need to take enough samples in order to account for all effects. If we want $n$ samples per oscillation we need to sample with a rate:
        \begin{equation}\label{eq:m4:methods:bessel_sampling}
            \Delta z = \frac{2\pi}{n_\mathrm{bessel}}.
        \end{equation}
        We will use $n_\mathrm{bessel}=25$ and the result may be seen in ~\cref{fig:m4:bessel}, from which we clearly see that we sample with enough accuracy. The first peaks occur at around the value of the order of the function. 
        \begin{figure}
            \includegraphics[width=\linewidth]{bessel.pdf}
            \caption{The spherical Bessel function for selected $l$-values.}
            \label{fig:m4:bessel}
        \end{figure}
        One ending note is that if the argument $z$ of the spherical Bessel functions are too large, some functions for finding it may break down, so proceed with caution. 

    \subsubsection{Line of sight integration}
        We are now ready to perform the actual line of sight integration from ~\cref{eq:m3:theory:line_of_sight_integral_definition}. We are supposed to integrate from $-\infty$ until today, but by investigating the integrand we may simplify this. ~\cref{fig:m4:LOS_integrand} show the integrand for a selected number of $l$-s. From the discussion of the source function in ~\cref{sec:m3:theory:line_of_sight}, we would expect most of its contribution to be around the time of recombination. This is emphasized in the plot, and we also see some effects at later times, up until today. These are the correction effects to the source function, which we need to take into account. The take home message is that there is very limited contribution from the time before recombination, so we choose some time right before recombination from which we start the integration. We must also here pay close attention to the sampling of the integrand because of the oscillations, and because it is a function of both $x$ and $k$. We sample w.r.t. $x$ as follows:
        \begin{equation}
            \Delta x = \frac{2\pi}{n_\mathrm{LOS}^{(x)}},
        \end{equation}
        and w.r.t. $k$ as:
        \begin{equation}
            \eta_0\Delta k = \frac{2\pi}{n_\mathrm{LOS}^{(k)}},
        \end{equation}
        where we use $n_\mathrm{LOS}^{(x)}=350$, and $n_\mathrm{LOS}^{(k)}=32$, since the integrand is a lot smoother in $k$ than $x$, we thus need a rather high resolution in $x$. 

        \begin{figure}
            \includegraphics[width=\linewidth]{LOS_integrand.pdf}
            \caption{Integrand of LOS integral, mostly governed by the source function. The contribution is very limited before, but peaked during recombination. We have some contribution between recombination and today due to the corrections to the source function. The plot shows (but indistinguishable) values for $k=k_\mathrm{min}$ and $k=k_\mathrm{max}$ in order to highlight the extreme effects.}
            \label{fig:m4:LOS_integrand}
        \end{figure}

        The result of the line of sight integral is the transfer function $\T_l(k, \eta=\eta_0)$, and can be seen for the same selected values of $l$ in ~\cref{fig:m4:transfer_function}. It appears that our sampling was successful as we have captured the oscillations. This is by far the most time-consuming part of the integration, and the result is thus splined for easy access later on. 
        \begin{figure}
            \includegraphics[width=\linewidth]{transfer_function.pdf}
            \caption{Transfer function after performing line of sight integration for a selected number of $l$-s. }
            \label{fig:m4:transfer_function}
        \end{figure}
    \subsubsection{Integrate across $k$}
        When integating across $k$ in order to obtain the power spectrum $C_l$ the main dependence on $k$ is through the transfer function, so we use the same resolution as earlier for $k$. 
        \begin{figure}
            \includegraphics[width=\linewidth]{C_l_integrand.pdf}
            \caption{ss}
            \label{fig:m3:some}
        \end{figure}


\subsection{Results}\label{sec:m4:results}
    ~\cref{fig:m4:angular_power_spectrum} shows the angular power spectrum as function of photon multipole $l$. The blue drawn line is the main power spectrum, while the dotted lines are the four constituents of the power spectrum that arise from the source function in ~\cref{eq:m3:theory:source_function}. The red lines are observations taken from ~\cite{Planck2020}, and the turquoise shade shows the theoretical cosmic variance as given in ~\cref{eq:m4:theory:cosmic_variance}. As expected, since we ignore both polarisation and neutrinos there is a discrepancy between the observed values and the theoretical prediction. This is most prominent for larger $l$-s where the observational constraints are low due to high statistical accuracy. In discussing this results, we will focus on the three main parts of the plots, namely the \textit{Sachs-Wolfe plateau} for low $l$-s, the \textit{acoustic oscillations} for intermediate $l$-s and the \textit{diffusion damping} for high $l$-s. Lastly we discuss the matter power spectrum in, where we have obtained the observational data from ~\cite{Chabanier_2019} and ~\cite{Hlozek_2012}.
    \begin{figure}
        \includegraphics[width=\linewidth]{power_spectrum.pdf}
        \caption{Angular power spectrum as function of photon multipole $l$. The blue line shows the angular power spectrum itself with the intrinsic cosmic variance overplotted in turquoise. The dotted lines are the individual effect of the different constituents of the source function. The red error bars are observational constraints.}
        \label{fig:m4:angular_power_spectrum}
    \end{figure}
    \subsubsection{The Sachs-Wolfe plateau}
        The Sachs-Wolfe plateau is the part of the angular power spectrum that appear fairly flat for low $l$, i.e. large scales, hence it names. These large scale represent modes that had not yet entered the horison at the time of recombination. According to the Sachs-Wolfe (SW) term in the source function in ~\cref{eq:m3:theory:source_function}, the major contribution to the power spectrum are the temperature anisotropies present at the last scattering surface. This is the main contributor to the transfer function and thus the power spectrum itself. These effects are described by the photon monopole, and the value of the gravitational perturbation, $\Psi$, whose main effect slowed the photons down through gravitational redshift.\footnote{Plus a small quadrupole correction.} For large scales, that have not yet entered the horison at the time of last scattering, the fluctuation to the gravitational perturbation have not yet been affected by causal physics. As a result, they closely resemble the initial perturbations induced by inflation. 

        Thus, the Sachs-Wolfe plateau may be understood as a tracer of the primordial power spectrum. If we continue to assume this take the form of a Harrison-Zel'dovich spectrum parametrised by the amplitdue $A_s$ and spectral index $n_s$ we are able to estimate the former by considering the amplitude of the Sachs-Wolfe plateau, and the latter by its shape. $n_s=1$ would generate a flat primordial power spectrum, whereas $n_s>1$ would give emphasis to small scales and vice versa. 

        We also take note of the large cosmic variance in this region. As already explained, this is and intrinsic feature of the angular power spectrum when comparing to observations, as we only have one universe to make observations in. The observational constraints are also significantly less accurate in this region, which is to be expected. 
        
        In figure ~\cref{fig:m4:angular_power_spectrum} we also see some contribution from the integrated Sachs-Wolfe (ISW) term, which occur due to changes in the gravitational potentials. Most prominent in the discussion of the Sachs-Wolfe plateau is the late-time ISW which occur around matter-dark-energy equality. These effects can be seen for the largest scales, as they happened quite recently, and contributes to the minor deviations away from a straight line for the Sachs-Wolfe plateau.
        
        There is also an increasing contribution of the Doppler term, which correspond to the peculiar velocity of matter (and our peculiar velocity relative to the CMB-frame). The Sachs-Wolfe plateau indicate the existence of large scale structures in the Universe, because of its power at the smallest $l$-s. It also provides valuable insight into the primordial power spectrum, and the process of inflation. 

    \subsubsection{Acoustic oscillations}
        The acoustic oscillations seen at intermediate scales in ~\cref{fig:m4:angular_power_spectrum} are a result of oscillations in the primordial plasma\footnote{Plasma containing tightly coupled baryons and photons, giving rise to the alternative name; \textit{baryon acoustic oscillations (BAO)}.} in the very early universe. These oscillations are a result of primordial fluctuations in the dark matter distribution, effectively creating gravitational wells in which baryonic matter accumulate. In areas where the density of baryonic matter is increased, so is the photon pressure as well, due to their tight coupling. The gravitational wells thus have an attractive effect on the primordial plasma while the photon pressure have a repulsive effect. The interplay between these two effect source acoustic oscillations that travel throughout the universe. 

        The physics of gravitational attraction and radiative repulsion are both causal, so the effects from these phenomena may only be conveyed within causally connected regions. Thus, the scale for which these oscillations may be observed is closely governed by the causal horison. Another thing to keep in mind is that this interplay between gravitational attraction and radiative repulsion require photons to be coupled to baryons. Therefore, we would think that the acoustic oscillations present at the last scattering surface is imprinted on the CMB and thus the angular power spectrum.

        The first peak at around $l\sim200$ represent the first contraction of the plasma. In order to have repulsion, there must first be attraction, thus this must have happened first. This contraction and subsequent rarefaction creates these oscillations in the primordial plasma, propagating with the speed of sound (found in ~\cref{eq:m2:theory:sound_speed}). The quantity measured in the angular power spectrum is the temperature fluctuations, not the oscillations themselves. However, the Sachs-Wolfe effect tells us that the radiation escaping from these oscillating fluctuations looses energy due to gravitational redshift. Anomalies in photon temperature are thus tracers of the potential wells from which these oscillations occur. Since most photons we observe today have free-streamed to us from the surface of last scattering, the peaks and troughs of the angular power spectrum represents the different scales affected by these propagating oscillations at the time of last scattering. The first peak represents the largest of these scale, i.e. the scale corresponding to the sound horison at the time of last scattering. In other word, it is the variation in photon temperature due to the very first contraction of the primordial fluid. 

        The second and third peak subsequently correspond to the first rarefaction and the second compression. These temperature fluctuations occur at smaller scales because they happened at a later time, leaving less time for the sound waves to propagate, resulting in a smaller angular scale at the time of recombination. In fact, it is fair to assume that the imprint of each scale happens at slightly different times, since the \textit{freeze-in} ultimately occur when the mean free path of photons becomes larger than the scale of the oscillating wave. However, in cosmic time these events are very close to each other, as the mean free path increased rapidly during recombination. Similar arguments follow for the next peaks, but these are greatly affected by damping. It is also worth noting that the acoustic oscillations seen in ~\cref{fig:m4:angular_power_spectrum} are mostly a result of the Sachs-Wolfe effect, i.e. the fluctuations of the gravitational potential at the time of last scattering. There is also a small contribution from the doppler term. \TODO{why?} 

    \subsubsection{The damping tail}
        In the primordial plasma that existed before recombination, photons were tightly coupled to baryons (matter) via Thomson scattering. This resulted in an opaque environment, where photons had a short mean free path, and energy was interchanged with every collision. These interactions and consequent energy transfers caused the temperature fluctuations at small scales to be smoothed. In terms of angles, it is fair to assume that the scattering of photons off electrons causes a dispersion across all angles, as none should be favoured. All of this results in an averaging of the temperature in local neighborhoods, effectively reducing the temperature fluctuations. The smaller the scale, the greater the effect. This phenomenon is called \textit{silk damping} or more generally \textit{diffusion damping}, ~\cite{dodelson2020modern}. In ~\cref{fig:m4:angular_power_spectrum} this can be observed as a damping tail towards the higher $l$-s, where the amplitude of the temperature fluctuations rapidly decrease, but still oscillating. The effect of the diffusion damping is thus to suppress the acoustic oscillations. 

    \subsubsection{Parameter dependence}
        So far we have studied how the angular power spectrum actually appear with our choice of parameters, but how would this appearance change if we changed any of the free parameters? Let's start with the curvature parameter $\O_{k0}$: In our fiducial cosmology, the universe is almost completely flat. If however, we had an open universe ($\O_{k0}>0$), the entire power spectrum would have been shifted towards the right, i.e. towards smaller scales. There are two ways of seeing this; firstly, for an open universe to exist, the acceleration of the expansion need to be large, in order to counteract and eventually rip apart the gravitational force. This faster expansion cools the universe down quicker, and thus recombination happens at an earlier time. Thus, the sound horison at this time must be smaller, and the fluctuations of the photons leaving it must also be present at a smaller scale. This corresponds to the first peak being present at larger $l$. From a geometrical point of view, and open universe 

    \subsubsection{The matter power spectrum}
    
    \begin{figure}
        \includegraphics[width=\linewidth]{matter_power_spectrum.pdf}
        \caption{Matter power spectrum as function of wavenumber $k$. The violet dotted line the equality scale, which is the wavenumber that correspond to the mode entering the horison at the time of radiation-matter equality. The red error bars are observational constraints.}
        \label{fig:m4:matter_power_spectrum}
    \end{figure}

\newpage
\section{Conclusion}\label{sec:conclusion}

Throughout this project we have investigated the unperturbed FLRW background and its properties, the epoch of recombination, and perturbation to the FLRW background. All of this enabled us to construct the angular power spectrum of the CMB and the matter power spectrum for the total density contrast; quantities which we were able to compare with actual physical observables. These predictions were quite accurate for large scales, but showing a discrepancy larger than the observational uncertainties for the small scales. This is because we ignore the effect of neutrinos, polarisation and heavier elements when evolving the perturbation equations. We also do not consider the epoch of reionisation. Including all these effect would be a natural next step in improving our model. However, we are able to qualitatively discuss the physics behind most aspects of both power spectrums, having succeeded in our initial aim of producing a pipeline that numerically calculated the CMB power spectrum and matter power spectrum, given some cosmological model.


%======================================================
%
%   End of document
%
%======================================================

\bibliography{references/ref}

% \appendix

% \section{Derivation of the Friedmann equations}\label{app:friedmann}
%     The Einstein equation relates the curvature of spacetime to the distribution of matter and energy within it, and is given by:

%     $$G_{\mu\nu} = 8\pi T_{\mu\nu}$$

%     where $G_{\mu\nu}$ is the Einstein tensor, which describes the curvature of spacetime, and $T_{\mu\nu}$ is the stress-energy tensor, which describes the distribution of matter and energy. To derive the Friedmann equations from this equation, we need to first make some assumptions about the geometry and matter content of the universe.

%     We assume that the universe is homogeneous and isotropic, which implies that the metric for the universe can be written in the following form:

%     $$ds^2 = -c^2 dt^2 + a(t)^2\left[\frac{dr^2}{1-kr^2} + r^2(d\theta^2 + \sin^2\theta d\phi^2)\right]$$

%     where $a(t)$ is the scale factor of the universe, $k$ is the curvature of space, and $c$ is the speed of light.

%     With this metric, we can compute the Christoffel symbols, which describe the connection coefficients of spacetime, using the following equation:

%     $$\Gamma^{\rho}{\mu\nu} = \frac{1}{2}g^{\rho\sigma}\left(\frac{\partial g{\sigma\mu}}{\partial x^{\nu}}+\frac{\partial g_{\sigma\nu}}{\partial x^{\mu}}-\frac{\partial g_{\mu\nu}}{\partial x^{\sigma}}\right)$$

%     where $g_{\mu\nu}$ is the metric tensor and $g^{\mu\nu}$ is its inverse. After computing all the non-zero Christoffel symbols, we can use them to calculate the components of the Einstein tensor, which are given by:

%     $$G_{00} = -3\frac{\ddot{a}}{a} - 3\frac{k}{a^2}$$
%     $$G_{ij} = (\ddot{a} + 2\frac{\dot{a}^2}{a^2} + 2\frac{k}{a^2})\delta_{ij}$$

%     where $\delta_{ij}$ is the Kronecker delta.

%     Next, we need to specify the stress-energy tensor, which describes the distribution of matter and energy in the universe. For a homogeneous and isotropic universe, this tensor takes the form of a perfect fluid, with energy density $\rho$ and pressure $p$ given by:

%     $$T_{00} = \rho c^2$$
%     $$T_{ij} = p a^2\delta_{ij}$$

%     Substituting these expressions for the stress-energy tensor into the Einstein equation, and equating the components of the Einstein tensor to the corresponding components of the stress-energy tensor, we obtain the following equations:

%     $$\frac{\ddot{a}}{a} = -\frac{4\pi G}{3}(\rho + 3p) + \frac{\Lambda}{3}$$
%     $$(\frac{\dot{a}}{a})^2 + \frac{k}{a^2} = \frac{8\pi G}{3}\rho + \frac{\Lambda}{3}$$

%     These are the Friedmann equations, which describe the evolution of the scale factor and energy density of the universe. The first equation describes the acceleration of the expansion of the universe, and the second equation relates the expansion rate to the energy density and curvature of space.


%     In order to compute the Christoffel symbols, we start with the metric for the universe, which we assume is homogeneous and isotropic:

%     $$ds^2 = -c^2 dt^2 + a(t)^2\left[\frac{dr^2}{1-kr^2} + r^2(d\theta^2 + \sin^2\theta d\phi^2)\right]$$

%     where $a(t)$ is the scale factor of the universe, $k$ is the curvature of space, and $c$ is the speed of light.

%     The non-zero components of the metric tensor are:

%     $$g_{00} = -c^2, \quad g_{11} = a^2\frac{1}{1-kr^2}, \quad g_{22} = a^2r^2, \quad g_{33} = a^2r^2\sin^2\theta$$

%     Using the metric tensor, we can calculate the inverse metric tensor:

%     $$g^{00} = -\frac{1}{c^2}, \quad g^{11} = \frac{1-kr^2}{a^2}, \quad g^{22} = \frac{1}{a^2r^2}, \quad g^{33} = \frac{1}{a^2r^2\sin^2\theta}$$

%     We can now use these expressions to compute the Christoffel symbols, which are given by:

%     $$\Gamma^0_{00} = \Gamma^0_{i0} = \Gamma^i_{00} = 0$$

%     $$\Gamma^i_{jk} = \frac{1}{2}g^{il}(\frac{\partial g_{jl}}{\partial x^k}+\frac{\partial g_{kl}}{\partial x^j}-\frac{\partial g_{jk}}{\partial x^l})$$

%     where $i,j,k,l$ are indices running over the three spatial coordinates, and $x^j$ are the coordinates themselves.

%     For the diagonal terms, we have:

%     $$\Gamma^1_{11} = -\frac{kr}{1-kr^2}, \quad \Gamma^2_{22} = -r(1-kr^2), \quad \Gamma^3_{33} = -r(1-kr^2)\sin^2\theta$$

%     For the off-diagonal terms, we have:

%     $$\Gamma^1_{22} = \Gamma^1_{33} = \frac{1}{r}, \quad \Gamma^2_{33} = -\sin\theta\cos\theta$$

%     All other Christoffel symbols are either zero or can be obtained by symmetry. We can now use these Christoffel symbols to calculate the components of the Einstein tensor, which are given by:

%     $$G_{00} = -3\frac{\ddot{a}}{a} - 3\frac{k}{a^2}$$
%     $$G_{ij} = (\ddot{a} + 2\frac{\dot{a}^2}{a^2} + 2\frac{k}{a^2})\delta_{ij}$$

%     where $\delta_{ij}$ is the Kronecker delta.

%     Next, we need to specify the stress-energy tensor, which describes the distribution of matter and energy in the universe. For a homogeneous and isotropic universe, this tensor takes the form of a perfect fluid, with energy density $\rho$ and pressure $p$ given by:

%     $$T_{00} = \rho c^2$$

%     $$T_{ij} = p a^2 \delta_{ij}$$

%     Plugging these expressions into the Einstein equation, $G_{\mu\nu} = \frac{8\pi G}{c^4}T_{\mu\nu}$, we get:

%     $$-3\frac{\ddot{a}}{a}-3\frac{k}{a^2}=\frac{8\pi G}{c^4}\rho c^2$$

%     $$(\ddot{a}+2\frac{\dot{a}^2}{a^2}+2\frac{k}{a^2})\delta_{ij}=-\frac{8\pi G}{c^4}p a^2 \delta_{ij}$$

%     Simplifying the second equation by dividing both sides by $\delta_{ij}$ and using the first equation to eliminate the term involving $k$, we get:

%     $$\ddot{a}+2\frac{\dot{a}^2}{a}-\frac{8\pi G}{3c^2}(\rho + \frac{3p}{c^2})=0$$

%     This is the first Friedmann equation, which describes the evolution of the scale factor of the universe. The second Friedmann equation can be obtained by taking the trace of the Einstein equation, which gives:

%     $$3\frac{\ddot{a}}{a}+3\frac{k}{a^2}=4\pi G(\rho+\frac{3p}{c^2})$$

%     Eliminating $k$ using the first Friedmann equation, we get:

%     $$\left(\frac{\dot{a}}{a}\right)^2=\frac{8\pi G}{3}\rho-\frac{kc^2}{a^2}$$

%     This is the second Friedmann equation, which relates the Hubble parameter (the time derivative of the scale factor) to the energy density of the universe.

%     Thus, we have derived the Friedmann equations from the Einstein equation by explicitly computing all the Christoffel symbols and using the stress-energy tensor of a perfect fluid.

\section{Useful derivations}\label{app:derivations}
    \subsection{Angular diameter distance}
        This is related to the physical distance of say, an object, whose extent is small compared to the distance at which we observe is. If the extension of the object is $\Delta s$, and we measure an angular size of $\Delta\theta$, then the angular distance to the object is:

        \begin{equation}\label{eq:app:derivations:angular_distance}
            d_A = \frac{\Delta s}{\Delta\theta} = \frac{\d s}{\d\theta} = \sqrt{e^{2x}r^2} = e^xr,
        \end{equation}
        where we inserted for the line element $\d s$ as given in equation \cref{eq:m1:theory:fundamentals:FLWR_line_element}, and used the fact that $\d t/\d \theta = \d r/\d\theta = \d\phi/\d\theta  = 0$ in polar coordinates. 

    \subsection{Luminosity distance}
        If the intrinsic luminosity, $L$ of an object is known, we can calculate the flux as: $F=L/(4\pi d_L^2)$, where $d_L$ is the luminosity distance. It is a measure of how much the light has dimmed when travelling from the source to the observer. For further analysis we observe that the luminosity of objects moving away from us is changing by a factor $a^{-4}$ due to the energy loss of electromagnetic radiation, and the observed flux is changed by a factor $1/(4\pi d_A^2)$. From this we draw the conclusion that the luminosity distance may be written as:
        \begin{equation}
            d_L = \sqrt{\frac{L}{4\pi F}} = \sqrt{\frac{d_A^2}{a^4}} = e^{-x}r 
        \end{equation}

    \subsection{Differential equations}
    From the definition of $e^x\d\eta = c\d t$ we have the following:
    \begin{equation}
        \begin{split}
            \dv{\eta}{t} &= \dv{\eta}{x}\dv{x}{t}= \dv{\eta}{x}H = e^{-x}c \\
            \implies \dv{\eta}{x} &= \frac{c}{\Hp}.
        \end{split}
    \end{equation}

    Likewise, for $t$ we have:
    \begin{equation}
        \begin{split}
            \dv{\eta}{t} &= \dv{\eta}{x}\dv{x}{t} = \dv{x}{t}\frac{c}{\Hp} = e^{-x}c\\
            \implies \dv{t}{x} &= \frac{e^x}{\Hp} = \frac{1}{H}.
        \end{split}
    \end{equation}


\section{Sanity checks}\label{app:sanity}
\subsection{For $\Hp$}
    We start with the Hubble equation from \cref{eq:m1:lambdaCDM:conformal_Hubble_equation} and realize that we may write any derivative of $U$ as
    \begin{equation}
        \dv[n]{U}{x} = \sum_i(-\alpha_i)^n\O_{i0}\expe{-\alpha_ix}.
    \end{equation}

    We further have:
    \begin{equation}
        \dv{\Hp}{x} = \frac{H_0}{2}U^{-\frac{1}{2}}\dv{U}{x},
    \end{equation}

    and
    
    \begin{equation}
        \begin{split}
            \dv[2]{\Hp}{x} &= \dv{}{x}\dv{\Hp}{x}\\
            &= \frac{H_0}{2}\left[\dv{U}{x}\left(\dv{}{x}U^{-\frac{1}{2}}\right) + U^{-\frac{1}{2}}\left(\dv{}{x}\dv{U}{x}\right)\right]\\
            &=H_0\left[\frac{1}{2U^{\frac{1}{2}}}\dv[2]{U}{x} - \frac{1}{4U^{\frac{3}{2}}}\left(\dv{U}{x}\right)^2\right]
        \end{split}
    \end{equation}


    Multiplying both equations with $\Hp^{-1} = 1/(H_0U^{\frac{1}{2}})$ yield the following:

    \begin{equation}
        \frac{1}{\Hp}\dv{\Hp}{x} = \frac{1}{2U}\dv{U}{x},
    \end{equation}

    and 

    \begin{equation}
        \begin{split}
            \frac{1}{\Hp}\dv[2]{\Hp}{x} &= \frac{1}{2U}\dv[2]{U}{x} - \frac{1}{4U^2}\left(\dv{U}{x}\right)^2 \\
            &= \frac{1}{2U}\dv[2]{U}{x}-\left(\frac{1}{\Hp}\dv{U}{x}\right)^2
        \end{split}
    \end{equation}

    We now make the assumption that one of the density parameters dominate $\O_i>> \sum_{j\neq i}\O_i$, enabling the following approximation:
    
    \begin{equation}
        \begin{split}
            U &\approx \O_{i0}\expe{-\alpha_ix} \\
            \dv[n]{U}{x} &\approx (-\alpha_i)^n\O_{i0}\expe{-\alpha_ix},
        \end{split}
    \end{equation}
    from which we are able to construct:
    \begin{equation}
        \frac{1}{\Hp}\dv{\Hp}{x} \approx \frac{-\alpha_i\O_{i0}\expe{-\alpha_ix}}{2\O_{i0}\expe{-\alpha_ix}} = -\frac{\alpha_i}{2},
    \end{equation}
    and
    \begin{equation}
        \begin{split}
            \frac{1}{\Hp}\dv[2]{\Hp}{x} &\approx \frac{\alpha^2\O_{i0}\expe{-\alpha_ix}}{2\O_{i0}\expe{-\alpha_ix}} - \left(\frac{\alpha_i}{2}\right)^2 \\
            &= \frac{\alpha_i^2}{2} - \frac{\alpha_i^2}{4} = \frac{\alpha_i^2}{4}
        \end{split}
    \end{equation}
    which are quantities which should be constant in different regimes and we can easily check if our implementation of $\Hp$ is correct, which is exactly what we sought. 

\subsection{For $\eta$}

    In order to test $\eta$ we consider the definition, solve the integral and consider the same regimes as above, where one density parameter dominates:

    \begin{equation}
        \begin{split}
            \eta &= \int_{-\infty}^x \frac{c\d x}{\Hp} = \frac{-2c}{\alpha_i}\int_{x=-\infty}^{x=x}\frac{\d\Hp}{\Hp^2} \\
            &= \frac{2c}{\alpha_i}\left(\frac{1}{\Hp(x)}-\frac{1}{\Hp(-\infty)}\right),
        \end{split}
    \end{equation}
    where we have used that:
    \begin{equation}
        \begin{split}
            \dv{\Hp}{x} &= -\frac{\alpha_i}{2}\Hp \\
            \implies \d x &= -\frac{2}{\alpha_i\Hp}\d\Hp.
        \end{split}
    \end{equation}
    
    Since we consider regimes where one density parameter dominates, we have that $\Hp(x)\propto \sqrt{\expe{-\alpha_ix}}$, meaning that we have:
    \begin{equation}
        \left(\frac{1}{\Hp(x)}-\frac{1}{\Hp(-\infty)}\right) \approx 
        \begin{cases}
            \frac{1}{\Hp} \quad &\alpha_i>0 \\
            -\infty \quad &\alpha_i <0.
        \end{cases}
    \end{equation}
    Combining the above yields:
    \begin{equation}
        \frac{\eta\Hp}{c} \approx 
        \begin{cases}
            \frac{2}{\alpha_i} \quad &\alpha_i>0 \\
            \infty \quad &\alpha_i<0.
        \end{cases}
    \end{equation}
    Notice the positive sign before $\infty$. This is due to $\alpha_i$ now being negative. 
    


\end{document}