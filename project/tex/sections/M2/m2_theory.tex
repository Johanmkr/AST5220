\subsection{Theory}\label{sec:m2:theory}
    Before recombination, the equilibrium between protons, electrons and photons is governed by the following interaction, from ~\cite{weinberg2008cosmology}\footnote{Where $H^*$ denotes excited states of hydrogen which will decay into neutral hydrogen.}:
    \begin{equation}\label{eq:m2:theory:equilibrium_interaction}
        e^-+p^+\leftrightharpoons H^* + \gamma,
    \end{equation}
    where a proton and an electron interact to form an excited hydrogen atom, which decays and emits a photon, or a photon excites and split a hydrogen atom into a free electron and a proton through \textit{Compton scattering}.\footnote{Elastic scattering of photons is technically Thomson scattering, but Compton scattering is a more general term and will be used (~\cite{dodelson2020modern}). This is also why we later use the Thomson cross section $\sigma_T$. The reaction is when a photon scatters of an electron, and possibly energises it enough to break out of the hydrogen atom, if already bound: $$\gamma + e^- \leftrightharpoons \gamma+e^-.$$} ~\cref{eq:m2:theory:equilibrium_interaction} is a reaction of the form $1+2\leftrightharpoons 3+4$, and we have from ~\cite{AST5220LectureNotes} that the Boltzmann equation for such a reaction is:
    \begin{equation}\label{eq:m2:theory:boltzmann_eq}
        \frac{1}{n_1e^{3x}}\dv{(n_1e^{3x})}{x} = -\frac{\Gamma}{H}\left(1-\frac{n_3n_4}{n_1n_2}\left(\frac{n_1n_2}{n_3n_4}\right)_\mathrm{eq}\right),
    \end{equation}
    where $n_i$ are the number densities of the reactants, $\Gamma$ is the reaction rate and $H$ the Hubble parameter (expansion rate of the universe). If the reaction rate is much larger than the expansion rate of the universe, $\Gamma \gg H$, then ~\cref{eq:m2:theory:equilibrium_interaction} ensures equilibrium between protons, electron and photons. When $\Gamma$ drops below $H$, then the expansion rate becomes dominant and the reaction rate is unable to sustain equilibrium. This happens when the temperature of the Universe becomes lower than the binding energy of hydrogen, hence stable neutral hydrogen is able to form.\footnote{Well, it is really not as simple, as neutral hydrogen is obtained from excited hydrogen and how this process go about is non-trivial. As we ignore re-ionisation, I will not delve into this. However, both ~\cite[p. 113-129]{weinberg2008cosmology}, ~\cite[p. 95-99]{dodelson2020modern} and ~\cite{AST5220LectureNotes} elaborate further on this.} As a consequence, the photons \textit{decouple} from the protons and electron. When $\Gamma \ll H$, there are practically no interactions and the number density becomes constant for a comoving volume. Massive particles \textit{freeze out} and their abundance become constant. 

\subsubsection{Hydrogen recombination}\label{sec:m2:theory:hydrogen_recombination}
    We express the electron density through the free electron fraction $X_e \equiv n_e/n_\mathrm{H} = n_e/n_b$ where we have assumed that hydrogen make up all the baryons ($n_b=n_\mathrm{H}$). We also ignore the difference between free protons and neutral hydrogen. From ~\cite{https://doi.org/10.48550/arxiv.astro-ph/0606683} we obtain:
    \begin{equation}\label{eq:m2:theory:baryon_number_density}
        n_b = \frac{\rho_b}{m_\mathrm{H}} = \frac{\O_b\rho_c}{m_\mathrm{H}}\expe{-3x},
    \end{equation}
    where $m_\mathrm{H}$ is the mass of the hydrogen atom, and $\rho_c$ the critical density today as defined earlier. Before recombination, no stable neutral hydrogen is formed, thus the electron and baryon density is the same, i.e. there are only free electrons so $X_e \simeq 1$. When in equilibrium, the r.h.s. of ~\cref{eq:m2:theory:boltzmann_eq} reduces to 0, which is called the \textit{Saha approximation}. The solution is in this regime described by the \textit{Saha equation}, which from ~\cite{dodelson2020modern} in physical units is:
    \begin{equation}\label{eq:m2:theory:Saha_equation}
        \frac{X_e^2}{1-X_e} = \frac{1}{n_b}\left(\frac{k_Bm_eT_b}{2\pi\hbar^2}\right)^{3/2}\expe{-\epsilon_0/k_BT_b},
    \end{equation}
    where $\epsilon_0 = 13.6\text{ eV}$ is the ionisation energy of hydrogen. The Saha equation is only a good approximation when $X_e \simeq 1$. Thus for $X_e < (1-\xi)$,\footnote{Where $\xi$ is some small tolerance, which have to be defined in some numerical model for when to abonden the Saha equation and use the more accurate, but computationally more expensive Peebles equation. This is typically $\xi=0.001$} which corresponds to the period during and after recombination, we have to make use of the more accurate \textit{Peebles equation}. From ~\cite{https://doi.org/10.48550/arxiv.astro-ph/0606683}:
    \begin{subequations}\label{eq:m2:theory:peebles_equation}
        \begin{equation}
            \dv{X_e}{x} = \frac{C_r(T_b)}{H}\left[\beta(T_b)(1-X_e)-n_\mathrm{H}\alpha^{(2)}(T_b)X_e^2\right],
            \tag{\ref{eq:m2:theory:peebles_equation}}
        \end{equation}
        where
        \begin{align}
                C_r(T_b) &= \frac{\Lambda_{2s-1s}+\Lambda_\alpha}{\Lambda_{2s-1s} + \Lambda_\alpha+\beta^{(2)}(T_b)},\label{eq:m2:theory:peebles_CR} \\
                \Lambda_{2s-1s} &= 8.227 \unit{s}^{-1}, \label{eq:m2:theory:peebles_lambda}\\
                \Lambda_\alpha &= \frac{1}{(\hbar c)^3}H\frac{(3\epsilon_0)^3}{(8\pi)^2n_{1s}}, \label{eq:m2:theory:peebles_lambda_alpha}\\
                n_{1s} &=(1-X_e)n_\mathrm{H}, \label{eq:m2:theory:peebles_ns}\\
                n_\mathrm{H} &= (1-Y_p)\frac{3H_0^2\O_{b0}}{8\pi Gm_\mathrm{H}}\expe{-3x}, \label{eq:m2:theory:peebles_nH}\\
                \beta^{(2)}(T_b) &= \beta(T_b)\expe{3\epsilon_0/4k_BT_b}, \label{eq:m2:theory:peebles_beta2}\\
                \beta(T_b) &= \alpha^{(2)}(T_b)\left(\frac{k_Bm_eT_b}{2\pi\hbar^2}\right)^{3/2}\expe{-\epsilon_0/k_BT_b}, \label{eq:m2:theory:peebles_beta}\\
                \alpha^{(2)}(T_b) &=\frac{\hbar^2}{c}\frac{64\pi}{\sqrt{27\pi}}\frac{\alpha^2}{m_e^2}\sqrt{\frac{\epsilon_0}{k_BT_b}}\phi_2(T_b), \label{eq:m2:theory:peebles_alpha}\\
                \phi_2(T_b) &= 0.448\ln\left(\frac{\epsilon_0}{k_BT_b}\right). \label{eq:m2:theory:peebles_phi2}
        \end{align}
    \end{subequations}

    The Peebles equation takes into account that the energy (excitation) of hydrogen formed through ~\cref{eq:m2:theory:equilibrium_interaction} vary, and that decays take place until we reach the $n=2$ level (first excited state), denoted by $^{(2)}$ in ~\cref{eq:m2:theory:peebles_CR}-~\cref{eq:m2:theory:peebles_phi2}. Recombination to the ground state is not relevant, as this leads to an ionised photon which immediately ionises a neutral hydrogen atom ~\cite[p. 97]{dodelson2020modern}. The $C_r$ is the probability that singly ionised hydrogen is reionised further, where $\beta^{(2)}$ and $\beta$ are the collisional ionisations from the first ionised state and ground state respectively. $\alpha^{(2)}$ is the recombination rate to excited states. For more detailed description of these terms, see ~\cite{Ma_1995}. \footnote{Because of this non-trivial path into the ground state, and the large photon to baryon number ratio, recombination happens later than when the temperature of the universe correspond to exactly the binding energy of neutral hydrogen (~\cite{https://doi.org/10.48550/arxiv.astro-ph/0606683})}

    We find $X_e$ by solving ~\cref{eq:m2:theory:Saha_equation} for $X_e > (1-\xi)$ and ~\cref{eq:m2:theory:peebles_equation} for $X_e < (1-\xi)$. In theory, it is possible to solve the Peebles equation at very early times, but the equation is very stiff resulting in unstable numerical solutions at early times (high temperatures), hence the Saha approximation.

\subsubsection{Visibility}\label{sec:m2:theory:visibility}
    Visibility is a concept tied to the optical depth and mean free path of a medium. The two latter are inversely proportional to each other. The mean free path is the average distance a photon travels before its direction is changed (often by scattering). Thus, a small mean free path gives results in a lot of collision across short distances, which occurs in optically thick media. The optical depth as a function of conformal time is defined as \cite{AST5220LectureNotes}:
    \begin{equation}\label{eq:m2:theory:optical_depth}
        \tau = \int_\eta^{\eta_0} n_e\sigma_\mathrm{T}\expe{-x}\d\eta',
    \end{equation}
    where $n_e$ is the electron density and $\sigma_\mathrm{T}$ is the Thompson cross-section. In differential form, restoring original units, this is:
    \begin{equation}\label{eq:m2:theory:optical_depth_differential}
        \dv{\tau}{x} = -\frac{cn_e\sigma_Te^x}{\Hp}.    
    \end{equation} 
    From this we define the visibility function, $g$:
    \begin{equation}\label{eq:m2:theory:visibility_function}
        \begin{split}
            g &= -\dv{\tau}{\eta}\expe{-\tau} = -\Hp\dv{\tau}{x}\expe{-\tau}\\
            \tilde{g} &\equiv -\dv{\tau}{x}\expe{-\tau} = \frac{g}{\Hp},
        \end{split}
    \end{equation}
    where $\tilde{g}$ is in terms of the preferred time variable, $x$. Notable thing about the visibility function $\tilde{g}$ is that it is a true probability distribution, describing the probability density of some photon to last have scattered at time $x$. Because of this, we have that $\int_{-\infty}^0\tilde{g}(x)\d x = 1$. We also take note of the derivative of the visibility function:
    \begin{equation}\label{eq:m2:theory:visibility_function_deriv}
        \dv{\tilde{g}}{x} = e^{-\tau}\left[\left(\dv{\tau}{x}\right)^2-\dv[2]{\tau}{x}\right]
    \end{equation}


\subsubsection{Sound horizon}
    Let's take a small step back and consider the situation of the early Universe. Before any decoupling, the photons and electrons are coupled through Thompson scattering, and protons and electrons are coupled through coulomb interactions. Because of this, photons interact with baryons and move alongside with them as one fluid, in which wave propagates with a speed $c_s$, from ~\cite{dodelson2020modern}:
    \begin{equation}\label{eq:m2:theory:sound_speed}
        c_s \equiv c\left[3(1+R)\right]^{-\frac{1}{2}} \quad ; \quad R\equiv\frac{3\O_b}{4\O_\gamma},
    \end{equation}
    where $R$ is the \textit{baryon-to-photon energy ratio}. By the definition of $R$, if the baryon density is negligible compared to the radiation density, $R\sim0$, and we recover the wave propagation speed in a relativistic fluid: $c_s=3^{-1/2}$ (~\cite{dodelson2020modern}). The total distance such a wave would have travelled in a time $t$ (since the beginning of the Universe) is called the \textit{sound horizon}, found by simply integrating $c_s$ through time, accounting for the expansion of space itself by including a factor $e^{-x}$:
    \begin{equation}\label{eq:m2:theory:sound_horizon_def}
        s = \int_0^tc_se^{-x}\d t = \int_{-\infty}^x\frac{c_s}{\Hp}\d x,
    \end{equation}
    where the variables are changed to $x$. On differential form:
    \begin{equation}\label{eq:m2:theory:sound_horizon_differential}
        \dv{s}{x} = \frac{c_s}{\Hp},
    \end{equation}
    which is a straightforward differential equation to solve given some initial conditions. 