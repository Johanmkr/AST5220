\subsection{Theory}\label{sec:m2:theory}
    Before recombination, the equilibrium between protons, electrons and photons is governed by the following interaction:
    \begin{equation}\label{eq:m2:theory:equilibrium_interaction}
        e^-+p^+\leftrightharpoons H + \gamma,
    \end{equation}
    where a proton and an electron interact to form an excited hydrogen atom, which decays and emits a photon, or a photon excites and split a hydrogen atom into an electron and a proton. This is a reaction of the form $1+2\leftrightharpoons 3+4$, and we have from ~\cite{AST5220LectureNotes} that the Boltzmann equation for such a reaction is:
    \begin{equation}\label{eq:m2:theory:boltzmann_eq}
        \frac{1}{n_1e^{3x}}\dv{(n_1e^{3x})}{x} = -\frac{\Gamma_1}{H}\left(1-\frac{n_3n_4}{n_1n_2}\left(\frac{n_1n_2}{n_3n_4}\right)_\mathrm{eq}\right),
    \end{equation}
    where $n_i$ are the number densities of the reactants, $\Gamma_1$ is the reaction rate and $H$ the Hubble parameter (expansion rate of the universe). If the reaction rate is much larger than the expansion rate of the universe, $\Gamma_1 \gg H$, then ~\cref{eq:m2:theory:equilibrium_interaction} ensures equilibrium between protons, electron and photons. When $\Gamma_1$ drops below $H$, then the expansion rate becomes dominant and the reaction rate is unable to sustain equilibrium. This happens when the temperature of universe becomes lower than the binding energy of hydrogen, hence stable neutral hydrogen is able to form. As a consequence, the photons \textit{decouple} from the protons and electron. When $\Gamma_1 \ll H$, there are practically no interactions and the number density becomes constant for a comoving volume. Massive particles \textit{freeze out} and their abundance become constant. 

\subsubsection{Free electron fraction $X_e$}\label{sec:m2:theory:free_electron_fraction}
    We express the electron density through the free electron fraction $X_e \equiv n_e/n_\mathrm{H} = n_e/n_b$ where we have assumed that hydrogen make up all the baryons ($n_b=n_\mathrm{H}$). We also ignore the difference between free protons and neutral hydrogen. From ~\cite{https://doi.org/10.48550/arxiv.astro-ph/0606683} we obtain:
    \begin{equation}\label{eq:m2:theory:baryon_number_density}
        n_b = \frac{\rho_b}{m_\mathrm{H}} = \frac{\O_b\rho_c}{m_\mathrm{H}}\expe{-3x},
    \end{equation}
    where $m_\mathrm{H}$ is the mass of the hydrogen atom, and $\rho_c$ the critical density today as defined earlier. Before recombination, no stable neutral hydrogen is formed, thus the electron and baryon density is the same, i.e. there are only free electrons so $X_e \simeq 1$. When in equilibrium, the r.h.s. of ~\cref{eq:m2:theory:boltzmann_eq} reduces to 0, which is called the \textit{saha approximation}. The solution is in this regime described by the \textit{Saha equation}, which from ~\cite{dodelson2020modern} in physical units is:
    \begin{equation}\label{eq:m2:theory:Saha_equation}
        \frac{X_e^2}{1-X_e} = \frac{1}{n_b}\left(\frac{k_Bm_eT_b}{2\pi\hbar^2}\right)^{3/2}\expe{-\epsilon_0/k_BT_b},
    \end{equation}
    where $\epsilon_0 = 13.6\text{ eV}$ is the ionisation energy of hydrogen. The Saha equation is only a good approximation when $X_e \simeq 1$. Thus for $X_e < (1-\xi)$,\footnote{Where $\xi$ is some small tolerance, which have to be defined in some numerical model for when to abonden the Saha equation and use the more accurate, but computationally more expensive Peebles equation. This is typically $\xi=0.001$} which corresponds to the period during and after recombination, we have to make use of the more accurate \textit{Peebles equation}. From ~\cite{https://doi.org/10.48550/arxiv.astro-ph/0606683}:
    \begin{subequations}\label{eq:m2:theory:peebles_equation}
        \begin{equation}
            \dv{X_e}{x} = \frac{C_r(T_b)}{H}\left[\beta(T_b)(1-X_e)-n_\mathrm{H}\alpha^{(2)}(T_b)X_e^2\right],
            \tag{\ref{eq:m2:theory:peebles_equation}}
        \end{equation}
        where
        \begin{align}
                C_r(T_b) &= \frac{\Lambda_{2s-1s}+\Lambda_\alpha}{\Lambda_{2s-1s} + \Lambda_\alpha+\beta^{(2)}(T_b)},\label{eq:m2:theory:peebles_CR} \\
                \Lambda_{2s-1s} &= 8.227 \unit{s}^{-1}, \label{eq:m2:theory:peebles_lambda}\\
                \Lambda_\alpha &= \frac{1}{(\hbar c)^3}H\frac{(3\epsilon_0)^3}{(8\pi)^2n_{1s}}, \label{eq:m2:theory:peebles_lambda_alpha}\\
                n_{1s} &=(1-X_e)n_\mathrm{H}, \label{eq:m2:theory:peebles_ns}\\
                n_\mathrm{H} &= (1-Y_p)\frac{3H_0^2\O_{b0}}{8\pi Gm_\mathrm{H}}\expe{-3x}, \label{eq:m2:theory:peebles_nH}\\
                \beta^{(2)}(T_b) &= \beta(T_b)\expe{3\epsilon_0/4k_BT_b}, \label{eq:m2:theory:peebles_beta2}\\
                \beta(T_b) &= \alpha^{(2)}(T_b)\left(\frac{k_Bm_eT_b}{2\pi\hbar^2}\right)^{3/2}\expe{-\epsilon_0/k_BT_b}, \label{eq:m2:theory:peebles_beta}\\
                \alpha^{(2)}(T_b) &=\frac{\hbar^2}{c}\frac{64\pi}{\sqrt{27\pi}}\frac{\alpha^2}{m_e^2}\sqrt{\frac{\epsilon_0}{k_BT_b}}\phi_2(T_b), \label{eq:m2:theory:peebles_alpha}\\
                \phi_2(T_b) &= 0.448\ln\left(\frac{\epsilon_0}{k_BT_b}\right). \label{eq:m2:theory:peebles_phi2}
        \end{align}
    \end{subequations}
    \TODO{Add $\sigma_T$ and $\alpha$ to nomenclature}.

    \TODO{Describe the above equations slightly}

    We find $X_e$ by solving ~\cref{eq:m2:theory:Saha_equation} for $X_e > (1-\xi)$ and ~\cref{eq:m2:theory:peebles_equation} for $X_e < (1-\xi)$. In theory, it is possible to solve the Peebles equation at very early times, but the equation is very stiff resulting in unstable numerical solutions at early times (high temperatures), hence the Saha approximation.

\subsubsection{Optical depth $\tau$, and visibility function $\tilde{g}$}\label{sec:m2:theory:optical_depth}
    The optical depth as a function of conformal time is defines as \cite{AST5220LectureNotes}:
    \begin{equation}\label{eq:m2:theory:optical_depth}
        \tau = \int_\eta^{\eta_0} n_e\sigma_\mathrm{T}\expe{-x}\d\eta',
    \end{equation}
    where $n_e$ is the electron density and $\sigma_\mathrm{T}$ is the Thompson cross-section. In differential form, restoring original units, this is:
    \begin{equation}\label{eq:m2:theory:optical_depth_differential}
        \dv{\tau}{x} = -\frac{cn_e\sigma_T}{H}.    
    \end{equation} 
    From this we define the visibility function, $g$:
    \begin{equation}\label{eq:m2:theory:visibility_function}
        \begin{split}
            g &= -\dv{\tau}{\eta}\expe{-\tau} = -\Hp\dv{\tau}{x}\expe{-\tau}\\
            \tilde{g} &\equiv -\dv{\tau}{x}\expe{-\tau} = \frac{g}{\Hp},
        \end{split}
    \end{equation}
    where $\tilde{g}$ is in terms of the preferred time variable, $x$. Notable thing about the visibility function $\tilde{g}$ is that it is a true probability distribution, describing the probability density of some photon to last have scattered at time $x$. Because of this, we have that $\int_{-\infty}^0\tilde{g}(x)\d x = 1$. We also take note of the derivative of the visibility function:
    \begin{equation}\label{eq:m2:theory:visibility_function_deriv}
        \dv{\tilde{g}}{x} = e^{-\tau}\left[\left(\dv{\tau}{x}\right)^2-\dv[2]{\tau}{x}\right]
    \end{equation}


\subsubsection{Sound horizon}
    One last thing we want to find is the sound horizon, 