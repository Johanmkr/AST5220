\subsection{Methods}\label{sec:m2:methods}

\subsubsection{Computing $X_e$}
    First things first, we need to compute the free electron fraction $X_e$. We are for the most part not interested in things happening in the future here, so the temporal range of choice will be $x\in[-20,0)$ where $x=0$ is today, and $x=-20$ is sufficiently long ago, so that the range encapsulated effect studied here. In the early Universe, the energies are so high that all baryonic matter is in the form of free electron, $X_e\simeq1$, so we will start by solving the Saha equation, ~\cref{eq:m2:theory:Saha_equation}. We continue to solve equation ~\cref{eq:m2:theory:Saha_equation} as long as $X_e>1-\xi$ where we use $\xi=0.01$.

    If we define:
    \begin{equation}
        K \equiv \frac{1}{n_b}\left(\frac{k_Bm_eT_b}{2\pi\hbar^2}\right)^{3/2}\expe{-\epsilon_0/k_BT_b},
    \end{equation}
    then equaton ~\cref{eq:m2:theory:Saha_equation} takes the form $X_e^2 + KX_e - K = 0$, which is solved as a normal quadratic equation\footnote{$ay^2+by+c=0$ has solutions $$y=\frac{-b\pm\sqrt{b^2-4ac}}{2}$$.}, where $a=1$, $b=K$ and $c=-K$. Since $0\leq X_e\leq1$ we choose the positive solution, given by: $X_e = (-K+\sqrt{K^2+4K})/2$. We note that for very large values of $K$, the two terms inside the parenthesis are \TODO{finish this ahah wtf}.


    We continue to solve the Peebles equation as stated in ~\cref{eq:m2:theory:peebles_equation}, where the r.h.s. is implemented sequentially as ~\cref{eq:m2:theory:peebles_CR}-~\cref{eq:m2:theory:peebles_phi2} in reverse order. The initial condition is the last computed electron fraction above the cut-off: $X_{e0}=\min(X_e>1-\xi)$ as found from the Saha equation. It is solved for the x-range not solved by Saha. 

    Having found $X_e$ for the entire x-range, we compute $n_e$ and spline both results.

\subsection{Computing $\tau$ and $\tilde{g}$}
    With $n_e$ we are able to solve the optical depth as defined in ~\cref{eq:m2:theory:optical_depth_differential}. The iniital condition for this equation is that the optical depth today is zero: $\tau(x=0)=0$, meaning we have to solve this backwards in time. This is done by using the negative differential:
    \begin{equation}
        \dv{\tau_\mathrm{rev}}{x_\mathrm{rev}} = -\dv{\tau}{x} = \frac{cn_e\sigma_Te^x}{\Hp},
    \end{equation}
    and solving for positive $x_\mathrm{rev}$: $x_\mathrm{rev}\in[0,20]$. In order to undo this reversal, we map $\tau=-\tau_\mathrm{rev}$ to its corresponding $x=-x_\mathrm{rev}$.



