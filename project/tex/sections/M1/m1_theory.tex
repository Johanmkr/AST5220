\subsection{Theory}\label{sec:m1:theory}

\subsubsection{Fundamentals}\label{sec:m1:theory:fundamentals}

    If we assume the universe to be homogeneous and isotropic, the line elements $\d s$ is given by the FLWR-metric as follows (in polar coordinates)\cite[eq. 1.1.11]{weinberg2008cosmology}:

    \begin{equation}\label{eq:m1:theory:fundamentals:FLWR_line_element}
        \d s^2 = -\d t^2 + e^{2x(t)}\left[ \frac{\d r^2}{1-kr^2}+r^2\d\theta^2 + r^2\sin^2\theta\d\phi^2 \right],
    \end{equation}
    where we have introduced $x(t) = \ln(a(t))$, the logarithm of the scale factor $a(t)$ \checkthis{include more (about k)} as our first measure of time. 

    We further model all forms of energy in the universe as perfect fluids, only characterised by their rest frame density $\rho$ and isotropic pressure $p$, and an equation of state relating the two:

    \begin{equation}\label{eq:m1:theory:fundamentals:equation_of_state}
        \omega=\frac{\rho}{p}.
    \end{equation}

    By conservation of energy and momentum we must satisfy $\nabla_\mu T^\munu=0$, which results in the following differential equations for the density \checkthis{include more here?} of each fluid $\rho_i$:

    \begin{equation}\label{eq:m1:theory:fundamentals:density_diff_eq}
        \dv{\rho_i}{t} +3H\rho_i(1+\omega_i) = 0,
    \end{equation}
    where we have introduced the Hubble parameter $H \equiv\dot{a}/a=\d x/\d t$. The solution to eq. \ref{eq:m1:theory:fundamentals:density_diff_eq} is of the form:

    \begin{equation}\label{eq:m1:theory:fundamentals:solution_to_density_diff_eq}
        \rho_i \propto e^{-3(1+\omega_i)x},
    \end{equation}
    where $\omega_\mathrm{M} = 0$ (matter), $\omega_\mathrm{rad}=1/3$ (radiation), $\omega_\Lambda=-1$ (cosmological constant) and $\omega_k=-1/3$ (curvature). 

    With these assumptions, the solution to the Einstein equations (eq. \ref{eq:introduction:einstein_equation}) are the Friedmann equations, the first of which describes the expansion rate of the universe:

    \begin{equation}
        \label{eq:m1:theory:fundamentals:first_friedmann_equation}
        H^2 = \frac{8\pi G}{3}\sum_i\rho_i - kc^2\expe{-2x} \\
    \end{equation}
    and the second describe how this expansion rate changes over time:
    \begin{equation}
        \label{eq:m1:theory:fundamentals:second_friedmann_equation}
        \dv{H}{t}+H^2 = -\frac{4\pi G}{3}\sum_i\left(\rho+\frac{3p}{c^2}\right).
    \end{equation}

    As of now, we are primarily interested in the first Friedmann equation. By introducing the critical density, $\rho_c\equiv2H^2/(8\pi G)$, we define the density parameters $\O_i=\rho_i/\rho_c$. We further define the density of the curvature $\rho_k\equiv-3kc^2\expe{-2x}/(8\pi G)$, which enables us to write eq. \ref{eq:m1:theory:fundamentals:first_friedmann_equation} as simply:

    \begin{equation}
        1=\sum_i\O_i,
    \end{equation}

    where the curvature density $\O_k$ is included in the sum. From \cref{eq:m1:theory:fundamentals:solution_to_density_diff_eq} we know the evolution of the densities in time, and if we assume the density values today, $\O_{i0}$, are known (or are free parameters), then eq. \ref{eq:m1:theory:fundamentals:first_friedmann_equation} may also be written as:

    \begin{equation}\label{eq:m1:theory:fundamentals:Hubble_equation}
        H = H_0\sqrt{\sum_i\O_{i0}\expe{-3(1+\omega_i)x}},
    \end{equation}

    which is the Hubble equation we will use further.
    \FIXME{references - use cref}




\subsubsection{Measure of time and space}\label{sec:m1:measure_time_space}
    The main measure of time is usually the scale factor $a$, or its logarithm $x$. We then have the \textit{cosmic time} $t$ defined as:

    \begin{equation}\label{eq:m1:theory:measures:cosmic_time}
        t = \int_0^a\frac{\d a}{aH} = \int_0^{e^x}\frac{\d x}{H}.
    \end{equation}
    Another temporal measure is the \textit{conformal time} $\eta$ defined as $\d t = e^x\d\eta$ yielding:

    \begin{equation}
        \eta = \int_0^{e^x}\frac{\d x}{e^xH} \equiv \int_0^{e^x}\frac{\d x}{\Hp},
    \end{equation}
    where $\Hp=e^xH$ is defined as the \textit{conformal Hubble parameter}. \checkthis{$\eta c$ yields the distance to the \textit{particle horizon}}. We may also choose to measure time in terms of the \textit{redshift} $z$, where $1+z = 1/a = \expe{-x}$.

    The comoving distance is defined as follows:

    \begin{equation}
        \chi = some integral = \eta_0-\eta
    \end{equation}
    \FIXME{check signs}

    \TODO{double check the integral limits}

    The radial distance is given in terms of the comoving distance and the curvature density today $\O_{k0}$ as:

    \begin{equation}\label{eq:m1:theory:measures:r_equation_def}
        r = \begin{cases}
            \chi \cdot \frac{\sin\left(\sqrt{\abs{\O_{k0}}}H_0\chi/c\right)}{\sqrt{\abs{\O_{k0}}}H_0\chi/c} \qquad &\O_{k0} < 0 \\
            \chi \qquad &\O_{k0} = 0 \\
            \chi \cdot \frac{\sinh\left(\sqrt{\abs{\O_{k0}}}H_0\chi/c\right)}{\sqrt{\abs{\O_{k0}}}H_0\chi/c} \qquad &\O_{k0} > 0
        \end{cases}
    \end{equation}
    
    It is then straigtforward to define the angular diameter distance:
    \begin{equation}\label{eq:m1:theory:measures:angular_distance_def}
        d_A = \expe{x}r,
    \end{equation}

    and the luminosity distance:
    
    \begin{equation}\label{eq:m1:theory:measures:luminosity_distance}
        d_L = \expe{-x}r.
    \end{equation}

    \TODO{derive the above?}

    We also have the following useful relations:
    \begin{equation}
        \label{eq:m1:theory:measures:useful_relations}
        \begin{split}
            \dv{\eta}{x}&=\frac{c}{\Hp}, \\
             \dv{t}{x} &= \frac{1}{H}.
        \end{split}
    \end{equation}
    \TODO{explain further above}




\subsubsection{$\Lambda$CDM-model}

    In the $\Lambda$CDM model, the universe consists of matter in terms of baryonic matter ($b$) and cold dark matter (CDM), radiation in terms of photons ($\gamma$) and neutrinos ($\nu$) and dark energy in terms of a cosmological constant ($\Lambda$). 



\begin{equation}\label{eq:m1:chi2_test_def}
    \chi^2(h, \O_{m0}, \O_{k0}) = \sum_{i=1}^N \frac{(d_L(z, \O_{m0}, \O_{k0}) - d_L^\mathrm{obs}(z_i))^2}{\sigma_i^2}
\end{equation}