\subsection{Theory}\label{sec:m1:theory}

\subsubsection{Fundamentals}\label{sec:m1:theory:fundamentals}

If we assume the universe to be homogeneous and isotropic, the line elements $\d s$ is given by the FLWR-metric as follows (in polar coordinates)\cite[eq. 1.1.11]{weinberg2008cosmology}:

\begin{equation}\label{eq:m1:theory:fundamentals:FLWR_line_element}
    \d s^2 = -\d t^2 + e^{2x(t)}\left[ \frac{\d r^2}{1-kr^2}+r^2\d\theta^2 + r^2\sin^2\theta\d\phi^2 \right],
\end{equation}
where we have introduced $x(t) = \ln(a(t))$, the logarithm of the scale factor $a(t)$ \checkthis{include more} as our first measure of time. 

We further model all forms of energy in the universe as perfect fluids, only characterised by their rest frame density $\rho$ and isotropic pressure $p$, and an equation of state relating the two:

\begin{equation}\label{eq:m1:theory:fundamentals:equation_of_state}
    \omega=\frac{\rho}{p}.
\end{equation}

By conservation of energy and momentum we must satisfy $\nabla_\mu T^\munu=0$, which results in the following differential equations for the density \checkthis{include more here?} of each fluid $\rho_i$:

\begin{equation}\label{eq:m1:theory:fundamentals:density_diff_eq}
    \dv{\rho_i}{t} +3H\rho_i(1+\omega) = 0,
\end{equation}
where we have introduced the Hubble parameter $H \equiv\dot{a}/a=\d x/\d t$. The solution to eq. \ref{eq:m1:theory:fundamentals:density_diff_eq} is of the form:

\begin{equation}\label{eq:m1:theory:fundamentals:solution_to_density_diff_eq}
    \rho_i \propto e^{-3(1+\omega_i)x},
\end{equation}
where $\omega_i$ takes different values depending on the fluid it models (more on this later). 

.... solution Friedmann eq... hubble equaiton.  



\subsubsection{Measure of time and space}\label{sec:m1:measure_time_space}

\subsubsection{$\Lambda$CDM-model}












The Hubble equation, where we allow for curvature is \checkthis{citation?}:
\begin{equation}\label{eq:m1:hubble_equation}
    \begin{split}
        H(x) = H_0\sqrt{\O_\mathrm{M0}\expe{-3x} + \O_\mathrm{R0}\expe{-4x} + \O_{k0}\expe{-2x} + \O_{\Lambda 0}},
    \end{split}
\end{equation}
where $\O_\mathrm{M0} = \O_{b0} + {\O_{\cdm0}}$ and $\O_{\mathrm{R}0} = \O_{\gamma 0} + \O_{\nu 0}$ are the present day values of the total matter and radiation densities. 

\checkthis{Something aboutu the curvature, and the evolution of density parameters here}.

\checkthis{derive?}
\begin{equation}\label{eq:m1:eta_diff_eq}
    \dv{\eta}{x} = \frac{c}{\Hp(x)}.
\end{equation}

\begin{equation}\label{eq:m1:t_diff_eq}
    \dv{t}{x} = \frac{1}{H(x)}.
\end{equation}

\begin{equation}\label{eq:m1:co_moving_distance_def}
    \chi(x) = \eta_0 -\eta(x).
\end{equation}

\begin{equation}\label{eq:m1:r_equation_def}
    r(\chi) = \begin{cases}
        \chi \cdot \frac{\sin\left(\sqrt{\abs{\O_{k0}}}H_0\chi/c\right)}{\sqrt{\abs{\O_{k0}}}H_0\chi/c} \qquad &\O_{k0} < 0 \\
        \chi \qquad &\O_{k0} = 0 \\
        \chi \cdot \frac{\sinh\left(\sqrt{\abs{\O_{k0}}}H_0\chi/c\right)}{\sqrt{\abs{\O_{k0}}}H_0\chi/c} \qquad &\O_{k0} > 0
    \end{cases}
\end{equation}

\begin{equation}\label{eq:m1:angular_distance_def}
    d_A(x) = \expe{x}r(\chi(x)).
\end{equation}

\begin{equation}\label{eq:m1:luminosity_distance_def}
    d_L = \expe{-x}r(\chi(x)).
\end{equation}


\begin{equation}\label{eq:m1:chi2_test_def}
    \chi^2(h, \O_{m0}, \O_{k0}) = \sum_{i=1}^N \frac{(d_L(z, \O_{m0}, \O_{k0}) - d_L^\mathrm{obs}(z_i))^2}{\sigma_i^2}
\end{equation}