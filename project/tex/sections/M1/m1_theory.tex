\subsection{Theory}\label{sec:m1:theory}

\subsubsection{Fundamentals}\label{sec:m1:theory:fundamentals}

    If we assume the universe to be homogeneous and isotropic, the line elements $\d s$ is given by the FLWR-metric, here in polar coordinates \cite[eq. 1.1.11]{weinberg2008cosmology}:
    \begin{equation}\label{eq:m1:theory:fundamentals:FLWR_line_element}
        \d s^2 = -\d t^2 + e^{2x}\left[ \frac{\d r^2}{1-kr^2}+r^2\d\theta^2 + r^2\sin^2\theta\d\phi^2 \right],
    \end{equation}
    where we have introduced $x = \ln(a)$ which will be our primary measure of time. 

    We further model all forms of energy in the universe as perfect fluids, only characterised by their rest frame density $\rho$ and isotropic pressure $p$, and an equation of state relating the two:
    \begin{equation}\label{eq:m1:theory:fundamentals:equation_of_state}
        \omega=\frac{\rho}{p}.
    \end{equation}
    By conservation of energy and momentum we must satisfy $\nabla_\mu T^\munu=0$, which results in the following differential equations for the density of each fluid $\rho_i$, from \cite{AST5220LectureNotes}:
    \begin{equation}\label{eq:m1:theory:fundamentals:density_diff_eq}
        \dv{\rho_i}{t} +3H\rho_i(1+\omega_i) = 0,
    \end{equation}
    where we have introduced the Hubble parameter $H \equiv\dot{a}/a=\d x/\d t$. The solution to \cref{eq:m1:theory:fundamentals:density_diff_eq} is of the form:
    \begin{equation}\label{eq:m1:theory:fundamentals:solution_to_density_diff_eq}
        \rho_i \propto e^{-3(1+\omega_i)x},
    \end{equation}
    where $\omega_\mathrm{M} = 0$ (matter), $\omega_\mathrm{rad}=1/3$ (radiation), $\omega_\Lambda=-1$ (cosmological constant) and $\omega_k=-1/3$ (curvature). 

    With these assumptions, the solution to the Einstein equations (\cref{eq:introduction:einstein_equation}) are the Friedmann equations, the first of which describes the expansion rate of the universe:
    \begin{equation}
        \label{eq:m1:theory:fundamentals:first_friedmann_equation}
        H^2 = \frac{8\pi G}{3}\sum_i\rho_i - kc^2\expe{-2x} \\
    \end{equation}
    and the second describe how this expansion rate changes over time:
    \begin{equation}
        \label{eq:m1:theory:fundamentals:second_friedmann_equation}
        \dv{H}{t}+H^2 = -\frac{4\pi G}{3}\sum_i\left(\rho+\frac{3p}{c^2}\right).
    \end{equation}
    As of now, we are primarily interested in the first Friedmann equation. By introducing the critical density, $\rho_c\equiv2H^2/(8\pi G)$, we define the density parameters $\O_i=\rho_i/\rho_c$. We further define the density of the curvature $\rho_k\equiv-3kc^2\expe{-2x}/(8\pi G)$, which enables us to write \cref{eq:m1:theory:fundamentals:first_friedmann_equation} as simply:
    \begin{equation}
        1=\sum_i\O_i,
    \end{equation}
    where the curvature density $\O_k$ is included in the sum. From \cref{eq:m1:theory:fundamentals:solution_to_density_diff_eq} we know the evolution of the densities in time, and if we assume the density values today, $\O_{i0}$, are known (or are free parameters), then \cref{eq:m1:theory:fundamentals:first_friedmann_equation} may also be written as:
    \begin{equation}\label{eq:m1:theory:fundamentals:Hubble_equation}
        H = H_0\sqrt{\sum_i\O_{i0}\expe{-3(1+\omega_i)x}},
    \end{equation}
    which is the Hubble equation we will use further.





\subsubsection{Measure of time and space}\label{sec:m1:measure_time_space}
    The main measure of time is usually the scale factor $a$, or its logarithm $x$. We then have the \textit{cosmic time} $t$ defined as:
    \begin{equation}\label{eq:m1:theory:measures:cosmic_time}
        t = \int_0^a\frac{\d a}{aH} = \int_{-\infty}^{x}\frac{\d x}{H}.
    \end{equation}
    Another temporal measure is the \textit{conformal time} $\eta$ defined as $c\d t = e^x\d\eta$ yielding:
    \begin{equation}\label{eq:m1:theory:measures:conformal_time}
        \eta = \int_0^a\frac{c\d a}{a^2H} = \int_{-\infty}^{x}\frac{c\d x}{e^xH} \equiv \int_{-\infty}^{x}\frac{c\d x}{\Hp},
    \end{equation}
    where $\Hp=e^xH$ is defined as the \textit{conformal Hubble parameter}. We may also choose to measure time in terms of the \textit{redshift} $z$, where $1+z = 1/a = \expe{-x}$. The comoving distance is defined as follows:
    \begin{equation}
        \label{eq:m1:theory:measures:conformal ditance}
        \chi = \int_1^a\frac{c\d a}{a^2H} = \int_0^x\frac{c\d x}{\Hp} = \eta_0-\eta
    \end{equation}
    The radial distance is given in terms of the comoving distance and the curvature density today $\O_{k0}$ as:
    \begin{equation}\label{eq:m1:theory:measures:r_equation_def}
        r = \begin{cases}
            \chi \cdot \frac{\sin\left(\sqrt{\abs{\O_{k0}}}H_0\chi/c\right)}{\sqrt{\abs{\O_{k0}}}H_0\chi/c} \qquad &\O_{k0} < 0 \\
            \chi \qquad &\O_{k0} = 0 \\
            \chi \cdot \frac{\sinh\left(\sqrt{\abs{\O_{k0}}}H_0\chi/c\right)}{\sqrt{\abs{\O_{k0}}}H_0\chi/c} \qquad &\O_{k0} > 0
        \end{cases}
    \end{equation}
    It is then straigtforward to define the angular diameter distance:
    \begin{equation}\label{eq:m1:theory:measures:angular_distance_def}
        d_A = \expe{x}r,
    \end{equation}
    and the luminosity distance:
    \begin{equation}\label{eq:m1:theory:measures:luminosity_distance}
        d_L = \expe{-x}r,
    \end{equation}
    both of which are derived in \cref{app:derivations}. The temporal quantities $\eta$ and $t$ have the following evolutions with $x$:
    \begin{equation}\label{eq:m1:theory:measures:eta_diffeq}
        \dv{\eta}{x} = \frac{c}{\Hp}.
    \end{equation}
    \begin{equation}\label{eq:m1:theory:measures:t_diffeq}
        \dv{t}{x} = \frac{1}{H}.
    \end{equation}
    Both differential equations are easy to solve numerically. Their derivation may also be found in \cref{app:derivations}

\subsubsection{$\Lambda$CDM-model}

    In the $\Lambda$CDM model, the universe consists of matter in terms of baryonic matter ($b$) and cold dark matter (CDM), radiation in terms of photons ($\gamma$) and neutrinos ($\nu$) and dark energy in terms of a cosmological constant ($\Lambda$). In addition, we must allow for some curvature ($k$). As a result, the parameters of the model will be the present values of the Hubble rate, $H_0$, the baryon density $\O_{b0}$, the cold dark matter density $\O_{\mathrm{CDM}0}$, photon density $\O_{\gamma0}$, neutrino density $\O_{\nu0}$, dark energy density $\O_{\Lambda0}$, and the curvature density $\O_{k0}$. The present temperature of the cosmic microwave background radiation $T_{\mathrm{CMB}0}$ fixes the radiation density today through:
    \begin{equation}\label{eq:m1:theory:lambdaCDM:radiation_densities}
        \begin{split}
            \O_{\gamma0} &= \frac{16\pi^3G}{90}\cdot\frac{(k_b T_{\mathrm{CMB}0})^4}{\hbar^3c^5H_0^2}, \\
            \O_{\nu0} &= N_\mathrm{eff} \cdot\frac{7}{8} \cdot \left( \frac{4}{3} \right)^{4/3}\cdot \O_{\gamma0}.
        \end{split}
    \end{equation}
    The total radiation density is $\O_\mathrm{rad}=\O_\gamma+\O_\nu$ and the total matter density is $\O_\mathrm{M} = \O_b+\O_\mathrm{CDM}$.

    The Hubble equation from \cref{eq:m1:theory:fundamentals:Hubble_equation} may be redefined in terms of the conformal Hubble parameter $\Hp$ as:
    \begin{equation}\label{eq:m1:lambdaCDM:conformal_Hubble_equation}
        \begin{split}
            \Hp &= H_0\sqrt{U}\\
            U &\equiv \sum_i\O_{i0}\expe{-\alpha_ix}, 
        \end{split}
    \end{equation}
    where we have defined $\alpha_i\equiv(1+3\omega_i)$ and $i\in\{\mathrm{M}, \mathrm{rad}, \Lambda, k\}$. Since we know the values of the various $\omega_i$ it follows that:
    \begin{equation}
        \label{eq:m1:theory:lambdaCDM:alpha_values}
        \begin{split}
            \alpha_\mathrm{M} &= 1\\
            \alpha_\mathrm{rad} &= 2\\
            \alpha_k &= 0\\
            \alpha_\Lambda &= -2
        \end{split}
    \end{equation}

\subsubsection{Equalities and present day values}\label{sec:m1:theory:equalities}
     Given the evolution of the density parameters with time, where the proportionality constant is the present day density, we introduce the \textit{radiation-matter equality}, i.e. the time radiation and matter densities were equal: $\rho_\mathrm{rad}=\rho_\mathrm{M}$. According to \cref{eq:m1:theory:fundamentals:solution_to_density_diff_eq} this can be expressed as:
     \begin{equation}\label{eq:m1:theory:equalities:radiation_matter}
        \begin{split}
            \rho_\mathrm{rad0}e^{-4x} &= \rho_\mathrm{M0}\expe{-3x} \\
            e^x &= \frac{\rho_\mathrm{rad0}}{\rho_\mathrm{M0}} \implies x_\mathrm{rM} = \ln\left(\frac{\O_\mathrm{rad0}}{\O_\mathrm{M0}}\right),
        \end{split}
     \end{equation}
     where $x_\mathrm{rM}$ now denotes the time of radiation-matter equality. 

     Similarly, the \textit{matter-dark energy equality}, where $\rho_\mathrm{M}=\rho_\Lambda$ can be found to be:
     \begin{equation}\label{eq:m1:theory:equalities:matter_dark_energy}
        \begin{split}
            \rho_\Lambda &= \rho_\mathrm{M0}\expe{-3x} \\
            \implies x_{\mathrm{M}\Lambda} &= \frac{1}{3}\ln\left(\frac{\O_\mathrm{M0}}{\O_\Lambda}\right)
        \end{split}
     \end{equation}
     The time of matter-dark energy equality coincides with when the universe starts to accelerate, since this acceleration is driven by the dark energy, represented by the cosmological constant. From this time onwards, dark energy dominates the universe, and thus accelerating the expansion. 

     The age of the universe today, and the conformal time today can both be found by evaluating the solutions to the differential equations of $t$ and $\eta$ at the present time (where $x=0$). This is done numerically. 

\subsubsection{Analytical solutions and sanity checks}\label{sec:m1:theory:sanity}
    There are several ways we may check that both our workings and numerical implementations are indeed correct. The simplest way is to always ensure that the sum of all density parameters add up to 1, for all times: $\sum_i\O_i=1$. 
    
    If we only consider the most dominant density parameter, that is $\O_i >> \sum_{j\neq i}\O_j$, we end up with the following analytical expressions for different temporal regimes:
    \begin{equation}
        \label{eq:m1:theory:sanity:first_deriv_sanity}
        \frac{1}{\Hp}\dv{\Hp}{x} \approx - \frac{\alpha_i}{2} = 
        \begin{cases}
            -1 \qquad &\alpha_\mathrm{rad} = 2 \\
            -\frac{1}{2} \qquad &\alpha_\mathrm{M} = 1\\
            1 \qquad &\alpha_\Lambda = -2
        \end{cases}
    \end{equation}
    \begin{equation}
        \label{eq:m1:theory:sanity:second_deriv_sanity}
        \frac{1}{\Hp}\dv[2]{\Hp}{x} \approx \frac{\alpha_i^2}{4} = 
        \begin{cases}
            1 \qquad &\alpha_\mathrm{rad} = 2 \\
            \frac{1}{4} \qquad &\alpha_\mathrm{M} = 1\\
            1 \qquad &\alpha_\Lambda = -2
        \end{cases}
    \end{equation}
    \begin{equation}
        \label{eq:m1:theory:sanity:eta_sanity}
        \frac{\eta\Hp}{c} \approx 
        \begin{cases}
            1 \qquad &\alpha_\mathrm{rad} = 2 \\
            2 \qquad &\alpha_\mathrm{M} = 1\\
            \infty \qquad &\alpha_\Lambda = -2
        \end{cases}
    \end{equation}
    These equations will be useful when making sure that the implementations are correct. For a thorough derivation, see \cref{app:sanity}.



