\subsection{Methods}\label{sec:m1:methods}

\subsubsection{Initial equation}
    We have to consider the time evolution of the density parameters, given some present value, as function of our chosen time parameter, here $x$. The density evolution is implemented as:
    \begin{equation}\label{eq:m1:methods:initial:density_evolution}
        \O_n = \expe{-\alpha_nx}\O_{n0}\Hp_\mathrm{rat}^2
    \end{equation}
    where we have defined the ratio $\Hp_\mathrm{rat} \equiv H_0/\Hp$, and the new index $n$ are all the densitis: $n\in\{b, \mathrm{CDM}, \gamma, \nu, \Lambda, k\}$.

    We also implement functions to solve for the luminosity distance (\cref{eq:m1:theory:measures:luminosity_distance}), angular distance (\cref{eq:m1:theory:measures:angular_distance_def}), and the conformal distance (\cref{eq:m1:theory:measures:conformal ditance}).


\subsubsection{ODEs and Splines}
    The differential equations for $\eta$ (\cref{eq:m1:theory:measures:eta_diffeq}) and $t$ (\cref{eq:m1:theory:measures:t_diffeq}) are solved numerically as ordinary differential equations with the Runge-Kutta 4 as advancement method. The equations are solved for $x\in(-20,5)$. As initial condition we would like $\eta(-\infty)$ which is obviously not possible to calculate, so we pick some very early time and use the analytical approximation in the radiation dominated era (\cref{eq:m1:theory:sanity:eta_sanity}), which yield:
    \begin{equation}\label{eq:m1:methods:odes:eta_initial}
        \eta(x_0) = \frac{c}{\Hp(x_0)}.
    \end{equation}
    Likewise for $t$, the initial condition is:
    \begin{equation}\label{eq:m1:methods:odes:t_initial}
        t(x_0) = \frac{1}{2H(x_0)}.
    \end{equation}
    \FIXME{WHY divided by 2?}
    
    

    We then proceed by making splines of both $\eta$ and $t$. 