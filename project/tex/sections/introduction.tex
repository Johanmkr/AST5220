\section*{Introduction}\label{sec:introduction}

The Cosmic Microwave Background radiation is the leftover radiation from the early universe. It is the most ancient light we can observe, having travelled towards us ever since the Universe became transparent. Therefore, it contains a significant amount of information and it is of great interest to understand why it looks the way it does. Why are there fluctuations in the CMB? In this project we take a closer look at the CMB power spectrum. This is a plot that show the distribution of temperature fluctuations in the CMB across different angular scaler. This is of great significance to us, since the CMB power spectrum is able to reveal information about the cosmological parameters of our universe, such as the various density components and Hubble constant. It is also able to tell us something about the large scale structures of the Universe, and the overall geometry of space itself. Also, which is perhaps the most interesting, it can yield information about the nature of dark energy. 

The overarching aim is to produce a pipeline that allows us to calculate numerically the CMB power spectrum given some cosmology. The steps will be presented chronologically, and we start by setting up the background cosmology in ~\cref{sec:m1}. Here we solve the evolutionary equations for an isotropic and homogeneous universe using the $\Lambda$CDM-model. One particularly important event in the evolution of the CMB is recombination, ultimately leading to photon decoupling. After this, the photons free stream towards us and is what we today see as the CMB. The entire ~\cref{sec:m2} is devoted to this event, and the time right before and right after it. In ~\cref{sec:m3} we take a step away from the isotropic and homogeneous universe and consider perturbations to the metric in conformal Newtonian gauge. The implications these metric perturbations have on the distribution of matter is discussed, and we end up with a set of coupled differential equations that we can solve numerically. The initial condition of these are found by considering the period of inflation in the very early Universe. 

\TODO{Intro to Milestone 4}
