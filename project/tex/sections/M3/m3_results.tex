\subsection{Results and discussion}\label{sec:m3:results}
    We present the results of the numerical solutions of the Einstein-Boltzmann equations and discuss them in due order starting with the potentials, then the multipoles and finally the matter perturbations. The main focus is on the evolution of these perturbations, leaving the discussion of the initial conditions for the next section. 

\subsubsection{Potentials}
    ~\cref{fig:m3:potentials} shows the metric perturbation potentials $\Phi$ and $\Psi$ as functions of time $x$ for the $k$-modes under investigation. The time of matter-radiation equality and recombination is marked in the plot as black dash-dotted and a shaded grey area respectively. Let's first consider the top panel, showing only $\Phi$. At early times, it is constant across all $k$-modes. \footnote{When referring to ``all $k$-modes'' it is implicit that we only mean the three modes considered here, but the qualitative discussions should be valid across all $k$-modes.} This is expected since at early times, the horizon is small and most modes are larger than this. Thus, they will be unaffected by causal physics and stay constant at their initial value. As time proceeds, the smaller $k$-modes will be surpassed by the horizon and are suddenly subjected to causal physics. We see from the top panel that if this happens in the radiation dominated regime (before radiation-matter equality), the potential will decline as $e^{-2x}$. This can be seen from ~\cref{eq:m3:theory:phi_perturbation}. However, for large scale modes, like the blue graph, where horizon crossing happens in the matter dominated regimes, then we expect the potential to decline with a factor $9/10$ as it transitions from radiation domination to matter domination. 
    
    The sum of the potentials in the bottom panel of ~\cref{fig:m3:potentials} goes as $\Psi+\Phi \sim e^{-2x}/k^2\T_2$, so it will closely follow the quadrupole term, but being suppressed by an exponential factor. Also, the $k^2$ term suggested larger value for small $k$-s (large scale). We save the discussion of the quadrupole, but the latter can clearly be seen as the large and intermediate scale modes show sinusoidal behaviour, with the large scale mode having a larger amplitude. Both are exponentially suppressed. These effects happen when the tight coupling epoch is over, since all multipoles except the monopole and dipole are suppressed during tight coupling. 
    \begin{figure}
        \includegraphics[width=\linewidth]{potentials.pdf}
        \caption{The metric perturbation potentials, $\Psi$ representing the Newtonian potential, and $\Phi$ representing the spatial curvature perturbation. Both panels show the evolution as function of the time $x$, for the three different $k$-modes outlined in ~\cref{sec:m3:methods}. The top panel shows $\Phi$ alone, while the bottom shows the sum of the two. The grey shaded are represent the time of recombinaiton as found in ~\cref{sec:m2}, highlighting the fact that it did not occur in an instant, and the dash-dotted black line is the time of radiation-matter equality as found in ~\cref{sec:m1}.}
        \label{fig:m3:potentials}
    \end{figure}

\subsubsection{Multipoles}
    We now focus our attention on the multipoles, starting with the monopole term expressed through the photon overdensity $\delta_\gamma = 4\T_0$. Mathematically, this relation can be seen  from either the parenthesis in ~\cref{eq:m3:theory:time_evolution_of_Phi} or the definition of $\mathcal{Y}$ in ~\cref{eq:m3:theory:metric_perturbations_final} following somewhat diffuse symmetry arguments. Physically it also makes sense since the photon monopole is some measure of the average photon temperature which intuitively can be though of as a photon overdensity. It can be seen for the various $k$-modes in ~\cref{fig:m3:monopole}, where we clearly see that the small scale perturbation undergoes horizon crossing first, and become subject to causal physics. This is manifest in the oscillation of the green curve in the figure. Metric perturbations and pressure will generate acoustic oscillations within causally connected regions. When the horizon increase, these oscillations will affect a spatially large area, affecting $k$-modes on larger and larger scales. This is also manifest in ~\cref{fig:m3:monopole} as the intermediate $k$-mode starts to oscillate later, and finally the large scale mode. 
    \begin{figure}
        \includegraphics[width=\linewidth]{monopole.pdf}
        \caption{Photon overdensity represented by the photon monopole for the various $k$-modes. The dashed black line is the time of recombination, and the dash-dotted black line is the time of radiation-matter equality.}
        \label{fig:m3:monopole}
    \end{figure}

    A similar discussion to the one above can be applied on the photon velocity $v_\gamma = -3\T_1$, as shown in ~\cref{fig:m3:dipole}. Similarly to the monopole, small scale modes enter the horizon first and starts to oscillate followed by larger scale modes. 

    \begin{figure}
        \includegraphics[width=\linewidth]{dipole.pdf}
        \caption{Photon velocity represented by the photon dipole for the various $k$-mdoes. term The dashed black line is the time of recombination, and the dash-dotted black line is the time of radiation-matter equality.}
        \label{fig:m3:dipole}
    \end{figure}

    The same is the case for the quadrupole in ~\cref{fig:m3:quadrapole}. However, as previously assumed, the quadrupole term is strongly suppressed during tight coupling, but behave similarly to the lower order multipoles after tight coupling. Considering the large and intermediate scale modes (blue and red curve) we are able to close the discussion of the sum of potential in ~\cref{fig:m3:potentials}, where the quadrupole was the main contributor to the sinusoidal frequency, and $k$ to the amplitude. 

    \begin{figure}
        \includegraphics[width=\linewidth]{quadrapole.pdf}
        \caption{Quadrapole term of the photon perturbation. This term only becomes relevant after tight coupling. The dashed black line is the time of recombination, and the dash-dotted black line is the time of radiation-matter equality.}
        \label{fig:m3:quadrapole}
    \end{figure}

\subsubsection{Matter perturbations}

    For the matter perturbations, we start with the overdensities for cold dark matter and baryons, shown in ~\cref{fig:m3:delta}. Intuitively, we would expect small fluctuations in density to occur at very small scales early on, and then at larger scales gradually as the horizon increase. During the radiation dominated regime, expect the matter densities to be relatively homogenous on large scales, but noticeable on small scaler. In the matter dominated regime we expect overdense regions to attract more and more matter, increasing the overdensities. These effect should be noticeable on larger scales even, i.e. the modes that undergo horizon crossing during matter dominatin. When inspecting ~\cref{fig:m3:delta} this is indeed what we observe. The overdensity of the small scale mode in green starts to increase during radiation domination, large scale mode in blue during matter domination and the intermediate scale mode in red somewhere in between. All modes increase during matter domination which is what we expected. During dark energy domination we would expect the acceleration of the Universe to start to break up some structures and decrease the horizon, which would also decrease the matter overdensities. Lastly, we take note of the oscillations in the baryon density right before and after matter-radiation equality, but save the discussion for later. 

    \begin{figure}
        \includegraphics[width=\linewidth]{delta.pdf}
        \caption{The overdensities of cold dark matter and barons for the various $k$-modes. The dashed black line is the time of recombination, and the dash-dotted black line is the time of radiation-matter equality.}
        \label{fig:m3:delta}
    \end{figure}

    The next thing to consider are the matter velocities, shown in ~\cref{fig:m3:velocity}
    \TODO{Comment these}

    \begin{figure}
        \includegraphics[width=\linewidth]{velocity.pdf}
        \caption{Velocities term}
        \label{fig:m3:velocity}
    \end{figure}

    The remaining question to answer now is why does the baryon overdensity and velocity oscillate for the modes entering the horizon during radiation domination? In order to answer this question we consider the largest mode among the three; $k=0.1$/Mpc, and plot the velocities and overdensities of the cold dark matter, baryons and photons in the same plot. The result is seen in ~\cref{fig:m3:velocity_comparison} for the velocities and in ~\cref{fig:m3:delta_comparison} for the overdensities. We have already made statements about the oscillation of the photon velocity from ~\cref{fig:m3:dipole}, and photon overdensity from ~\cref{fig:m3:monopole}. The most important time in ~\cref{fig:m3:velocity_comparison} and ~\cref{fig:m3:delta_comparison} is the time of recombination (dashed line), which is closely related to the time of decoupling and last scattering, when the photons decouple from the baryons. Before this, baryons and photons are tightly coupled so we expect them to evolve similarly. It therefore makes sense that we have oscillations in the baryon velocity and overdensity before recombinations, as the baryons were tightly coupled to the photons. After decoupling, the baryon velocity and overdensity show similar behaviour to those of cold dark matter. At the same time, the oscillations of the photon monopole and dipole causes the photon velocity and overdensity to gradually decrease and converge as the photons free stream through the Universe. 

    \begin{figure}
        \includegraphics[width=\linewidth]{velocity_comparison.pdf}
        \caption{Velocities term}
        \label{fig:m3:velocity_comparison}
    \end{figure}

    \begin{figure}
        \includegraphics[width=\linewidth]{delta_comparison.pdf}
        \caption{Velocities term}
        \label{fig:m3:delta_comparison}
    \end{figure}

