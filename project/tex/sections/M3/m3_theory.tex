\subsection{Theory}\label{sec:m3:theory}

\subsubsection{Metric perturbations}
    The perturbed metric in the conformal-Newtonian gauge is given in ~\cite{https://doi.org/10.48550/arxiv.astro-ph/0606683} as:
    \begin{equation}\label{eq:m3:theory:perturbed_metric}
        g_\munu = \begin{pmatrix}
            -(1+2\Psi) & 0 \\
            0 & e^{2x}\delta_{ij}(1+2\Phi)
        \end{pmatrix}
    \end{equation}
    This means that we perturb the FLRW-metric with $\Psi\ll1$ corresponding to the Newtonian potential governing the motion of non-relativistic particles and $\Phi\ll1$ governing the perturbation of the spatial curvature. \footnote{$\Phi$ may also be interpreted as a \textit{local perturbation to the scale factor}, ~\cite{dodelson2020modern}.} The comoving momentum in this spacetime is:
    \begin{equation}
        P^\mu = \left[E(1-\Psi), p^i\frac{1-\Phi}{a}\right].
    \end{equation}
    By considering this momentum, and the geodesic equation in this perturbed spacetime we obtain the following ~\cite[Eqs. 3.62, 3.69, 3.71]{dodelson2020modern}:
    \begin{subequations}\label{eq:m3:theory:geodesic_results}
        \begin{align}
            \dv{x^i}{t} &= \frac{\hat{p}^i}{a}\frac{p}{E}(1-\Phi+\Psi) \label{eq:m3:theory:metric_perturb_dxidt}\\
            \dv{p^i}{t} &= -\left(H+\dv{\Phi}{t}\right)p^i-\frac{E}{a}\pdv{\Phi}{x^i}-\frac{1}{a}\frac{p^i}{E}p^k\pdv{\Phi}{x^k} + \frac{p^2}{aE}\pdv{\Phi}{x^i} \label{eq:m3:theory:metric_perturb_dpidt}\\
            \dv{p}{t} &= -\left(H+\dv{\Phi}{t}\right)p-\frac{E}{a}\hat{p}^i\pdv{\Psi}{x^i} \label{eq:m3:theory:metric_perturb_dpdt}
        \end{align}
    \end{subequations}
    Inserting ~\cref{eq:m3:theory:geodesic_results} into ~\cref{eq:m2:theory:bolzmann_expanded}, and for now assuming $C[f]=0$ yield the \textit{collisionless Bolzmann equations}. Keeping terms to first order only,\footnote{This is justified by the ansatz that deviations away from the equilibrium distribution of radiation in the inhomogeneous universe are of same order as the spacetime perturbations $\Phi$ and $\Psi$, ~\cite{dodelson2020modern}.} yield the collisionless Bolzmann equation: ~\cite[Eq. 3.83]{dodelson2020modern}:
    \begin{equation}\label{eq:m3:theory:collisionless_bolzmann_general}
        \dv{f}{t} = \pdv{f}{t}+\frac{p}{E}\frac{\hat{p}^i}{a}\pdv{f}{x^i}-\left[H+\dv{\Phi}{t} + \frac{E}{ap}\hat{p}^i\pdv{\Psi}{x^i}\right]p\pdv{f}{p}.
    \end{equation}
    Future work consists mainly of evaluating the collision terms for each species and equate it to ~\cref{eq:m3:theory:collisionless_bolzmann_general}

\subsubsection{Fourier space and multipole expansion}
    Consider a function $f(\vec{x},t)$. Its Fourier transform $\FF$ and inverse $\FF^{-1}$ are defined as:
    \begin{equation}\label{eq:m3:theory:fourier_def}
            \FF[f(\vec{x},t)]  \equiv \frac{1}{(2\pi)^{3/2}}\int e^{-i\vec{k}\cdot\vec{x}}f(\vec{x},t)\d^3 x = \tilde{f}(\vec{k},t),
    \end{equation}
    \begin{equation}\label{eq:m3:theory:fourier_inv_def}
        \FF^{-1}[\tilde{f}(\vec{k},t)] \equiv \frac{1}{(2\pi)^{3/2}}\int e^{i\vec{k}\cdot\vec{x}}\tilde{f}(\vec{k},t)\d^3 k = f(\vec{x},t).
    \end{equation}
    It becomes apparent from these definitions that taking the spatial derivative with respect to $\vec{x}$ in real space, is the same as multiplying the function with $i\vec{k}$ in Fourier space. This leads to the following property: $\FF[\nabla f(\vec{x},t)] = i\vec{k}\FF[f(\vec{x},t)]$. This is of major significance when working with partial differential equations (PDEs), where:
    \begin{equation}\label{eq:m3:theory:fourier_pde_tricks}
        \begin{split}
            \FF[\nabla^2f(\vec{x},t)] &= i^2\vec{k}\cdot\vec{k}\FF[f(\vec{x},t)] = -k^2\FF[f(\vec{x},t)]\\
            \FF\left[\dv[n]{f(\vec{x},t)}{t}\right] &= \dv[n]{t}\FF[f(\vec{x},t)].
        \end{split}
    \end{equation} 
    The two equations in ~\cref{eq:m3:theory:fourier_pde_tricks} have the ability of reducing PDEs down to a set of decoupled ODEs. This means that we are able to solve for each mode $k=\abs{\vec{k}}$ independently, which will be of great impact for the equations to come. 

    We will also work with multipole expansions, which are series written as sums of \textit{Legendre polynomials} expanded in $\mu=\cos{\theta}\in[-1,1]$ as:
    \begin{equation}\label{eq:m3:theory:legendre_expansion}
        f(\mu) = \sum_{l=0}^{\infty}f_l\mathcal{P}_l(\mu),
    \end{equation}
    where $\mathcal{P}_l$ is the $l$-th Legendre polynomial. These are orthogonal in such a way that they form a complete basis, enabling us to express any $f(\mu)$ as in ~\cref{eq:m3:theory:legendre_expansion}. The coefficients $f_l$ are the \textit{Legendre multipoles}:
    \begin{equation}\label{eq:m3:theory:legendre_coeff}
        f_l = \frac{2l+1}{2}\int_{-1}^1f(\mu)\mathcal{P}_l(\mu)\d\mu.
    \end{equation}




\subsubsection{Einstein-Boltzmann equations}
    We have two perturbations to the metric, $\Phi(\vec{x}, t)$ to the spatial curvature, and $\Psi(\vec{x},t)$ to the Newtonian potential. We seek to find the effect of these perturbations on baryonic matter, dark energy and radiation, as they ``live'' in a now perturbed spacetime. Let's start by defining the perturbation to the photons, $\T(\vec{x}, \hat{\vec{p}}, t)$, to be the variation of photon temperature around an equilibrium temperature $T^{(0)}$:
    \begin{equation}\label{eq:m3:theory:temperature_perturbation}
        T(\vec{x}, \hat{\vec{p}}, t) = T^{(0)}\left[1+\T(\vec{x}, \hat{\vec{p}}, t)\right].
    \end{equation}
    This is dependent on the location $\vec{x}$ and the direction of propagation $\hat{\vec{p}}$, thus capturing both inhomogeneities and anisotropies. We assume $\T$ to be independent of the momentum magnitude.\footnote{This follows from the fact that the magnitude of the photon momentum is virtually unchanged by the dominant form of interaction, Compton scattering ~\cite{dodelson2020modern}.}
    The collision terms for the photons are governed by Compton scattering. We use the form found in ~\cite[Eq. 5.22]{dodelson2020modern}\TODO{assumptions: ignore polarisation, and angular dep. of thomson cross sec}:
    \begin{equation}\label{eq:m3:theory:photon_collision_term}
        C[f(\vec{p})] = -p^2\pdv{f^{(0)}}{p}n_e\sigma_T[\T_0 - \T(\hat{\vec{p}}) + \hat{\vec{p}}\cdot\vec{v}_b]
    \end{equation}
    where $\T_0$ is the monopole term.\footnote{This is the integral over the photon perturbation at any given point, over all photon directions. It is given by $$\T_0(\vec{x}, t) \equiv \frac{1}{4\pi}\int \d\Omega'\T(\vec{x}, \hat{\vec{p}}', t)$$ where $\O'$ is the solid angle spanned by $\hat{\vec{p}}'$ ~\cite{dodelson2020modern}.} $\vec{v}_b$ is the bulk velocity of the electrons involved in the process, and is the same as for baryons, hence the subscript. The distribution function for radiation follows the Bose-Einstein distribution function, so we expand $f$ around its zeroth order Bose-Einstein form, ~\cite[Eq. 5.2-5.9]{dodelson2020modern}, using the temperature perturbation in ~\cref{eq:m3:theory:temperature_perturbation}\TODO{Include equation 5.9 in Dodelson?}. This is then inserted into ~\cref{eq:m3:theory:collisionless_bolzmann_general}, which we equate to the collision term in ~\cref{eq:m3:theory:photon_collision_term} in order to obtain the following full Boltzmann equation for radiation:\footnote{Where of course $m=0\iff E=p$.}
    \begin{equation}\label{eq:m3:theory:boltzmann_equation_radiation_full}
        \dv{\T}{t} + \frac{\hat{p}^i}{a}\pdv{\T}{x^i} + \dv{\Phi}{t} + \frac{\hat{p}^i}{a}\pdv{\Psi}{x^i} = n_e\sigma_T\left[\T_0-\T+\vec{\hat{p}}\cdot\vec{v}_b\right]
    \end{equation} 
    
    For massive particles, we start with cold dark matter (CDM). Firstly, we assume cold dark matter to not interact with any other species, nor self-interact. Thus, we do not have any collision terms. Further, we also assume it to behave like a fluid, neglecting any terms not to first order. We consider cold dark matter to be non-relativistic, thus they will only have a sizeable monopole and dipole term, which means that the evolution is fully characterised by the density and velocity, ~\cite{AST5220LectureNotes}. \TODO{how much about moments should I explain?} Therefore, we take the first and second moment of ~\cref{eq:m3:theory:collisionless_bolzmann_general} and consider them to first order, in order to retrieve the cosmological generalisation of the continuity equation ~\cite[Eq. 5.41]{dodelson2020modern}:
    \begin{equation}\label{eq:m3:theory:cdm_continuity_equation}
        \pdv{n_c}{t}+\frac{1}{a}\pdv{(n_cv_c^i)}{x^i} + 3\left[H+\pdv{\Phi}{t}\right]n_c=0,
    \end{equation}
    and the Euler equation ~\cite[Eq. 5.50]{dodelson2020modern}:
    \begin{equation}\label{eq:m3:theory:cdm_euler_equation}
        \pdv{v_c^i}{t} + Hv_c^i + \frac{1}{a}\pdv{\Psi}{x^i} = 0.
    \end{equation}
    In both ~\cref{eq:m3:theory:cdm_continuity_equation} and ~\cref{eq:m3:theory:cdm_euler_equation}, $n_c$ is the cold dark matter number density, $\vec{v}_c$ its bulk velocity. We then consider the perturbation of $n_c$ to first order:
    \begin{equation}\label{eq:m3:theory:cdm_number_density_perturbation}
        n_c(\vec{x}, t) = n_c^{(0)}[1+\delta_c(\vec{x},t)],
    \end{equation}
    and consider the first order perturbation to ~\cref{eq:m3:theory:cdm_continuity_equation}:
    \begin{equation}\label{eq:m3:theory:cdm_density_perturbation}
        \pdv{\delta_c}{t}+\frac{1}{a}\pdv{v_c^i}{x^i} + 3\pdv{\Phi}{t} = 0.
    \end{equation}
    ~\cref{eq:m3:theory:cdm_density_perturbation} and ~\cref{eq:m3:theory:cdm_euler_equation} now described the evolution of the density perturbation $\delta_c$ and bulk velocity $\vec{v}_c$ of cold dark matter.

    For baryons (protons and electrons) we also assume them to behave like a non-relativistic fluid, so taking moments is a similar task as for cold dark matter. The only difference is that baryons interact with each other through Coulomb scattering and Compton scattering. We may ignore Compton scattering between protons and photons due to the small cross-section, but electrons are coupled to both photons and protons. Since the first moment of the Boltzmann equation represents conservations of particle number, and none of the above interactions changes the total baryon particle number, the continuity equation is identical to ~\cref{eq:m3:theory:cdm_continuity_equation}, but for baryons. We also have a baryon perturbation similar to ~\cref{eq:m3:theory:cdm_number_density_perturbation}, which altogether results in the following density perturbation for baryons:
    \begin{equation}\label{eq:m3:theory:baryon_density_perturbation}
        \pdv{\delta_b}{t}+\frac{1}{a}\pdv{v_b^i}{x^i} + 3\pdv{\Phi}{t} = 0.
    \end{equation}
    For the Euler equation, we now have to consider the collision terms, where momentum is conserved, but transferred between the baryons and photons. This collision term is found from considering the first moment of the photon distribution and find the momentum transfer due to Compton scattering. According to ~\cite{AST5220LectureNotes} the momentum transfer in the baryon equation is $-n_e\sigma_TR^{-1}(v_\gamma^i-v_b^i)$


    \TODO{check both above and below}
    
    \begin{equation}
        \pdv{v_b^i}{t}+Hv_b^i + \frac{1}{a}\pdv{\Psi}{x^i} = -n_e\sigma_TR^{-1}\left[\T_1+\frac{1}{3}v_b^i\right]
    \end{equation}

    |

    |
    
    |
    
    |
    
    |
    
    |
    
    |
    
    |

     
    \begin{subequations}
        \begin{align}
            \dot{\T} &= -ik\mu(\T+\Psi)-\dot{\Phi}-\dot{\tau}\left[\T_0-\T+i\mu v_b-\frac{\mathcal{P}_2\T_2}{2}\right],\\
            \dot{\delta}_\cdm &= -3\dot{\Phi}+kv_\cdm \\
            \dot{v}_\cdm &= -k\Psi-\Hp v_\cdm \\
        \end{align}
    \end{subequations}

    
    Photon temperature multipoles
    \begin{subequations}\label{eq:m3:theory:photon_temp_multipoles}
        \begin{align}
            \T_0' &= -\frac{ck}{\Hp}\T_1-\Phi',\label{eq:m3:theory:photon_monopole}\\
            \T_1' &= \frac{ck}{3\Hp}\T_0 - \frac{2ck}{3\Hp}\T_2 +\frac{ck}{3\Hp}\Psi+\tau'\left[\T_1+\frac{1}{3}v_b\right],\label{eq:m3:theory:photon_dipole}\\
            \T_l &= \begin{cases}
                \frac{lck\T_{l-1}}{(2l+1)\Hp} -\frac{(l+1)ck\T_{l+1}}{(2l+1)\Hp} +\tau'\left[\T_l-\frac{\T_2}{10}\delta_{l,2}\right] , \quad &l\geq2\\
                \\
                \frac{ck\T_{l-1}}{\Hp} - c\frac{(l+1)\T_l}{\Hp\eta}+\tau'\T_l, &l=l_\mathrm{max}
            \end{cases}\label{eq:m3:theory:photon_multipole}
        \end{align}
    \end{subequations}


    
    Cold dark matter and baryons
    \begin{subequations}\label{eq:m3:theory:cdm_baryon_diffeqs}
        \begin{align}
            \delta'_\cdm &= \frac{ck}{\Hp}v_\cdm-3\Phi', \label{eq:m3:theory:delta_cdm}\\
            v'_\cdm &= -v_\cdm-\frac{ck}{\Hp}\Psi, \label{eq:m3:theory:v_cdm}\\
            \delta'_b &= \frac{ck}{\Hp}v_b - 3\Phi', \label{eq:m3:theory:delta_b}\\
            v'_b &= -v_b - \frac{ck}{\Hp} + \tau'R^{-1}(3\T_1+v_b) \label{eq:m3:theory:v_b}
        \end{align}
    \end{subequations}
    where $R$ is defined in ~\cref{eq:m2:theory:sound_speed}


    Metric perturbations

    \begin{subequations}\label{eq:m3:theory:metric_perturbations_final}
        \begin{align}
            \Phi' &= \Psi - \frac{c^2k^2}{3\Hp^2}\Phi + \frac{H_0^2}{2\Hp^2}\mathcal{Y}, \label{eq:m3:theory:phi_perturbation}\\
            \Psi &= -\Phi - \frac{12H_0^2}{c^2k^2}\O_\gamma\T_2. \label{eq:m3:theory:psi_perturbation}
        \end{align}
    \end{subequations}
    where $\mathcal{Y} = \O_\cdm\delta_\cdm+\O_b\delta_b+4\O_\gamma\T_0$

\subsubsection{Tight coupling regime}

\subsubsection{Inflation}
\subsubsection{Initial conditions}
